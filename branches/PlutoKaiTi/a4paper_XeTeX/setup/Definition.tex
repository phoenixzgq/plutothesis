% -*-coding: utf-8 -*-
%%%%%%%%%%%%%%%%%%%%%%%%%%%%%%%%%%%%%%%%%%%%%%%%%%%%%%%%%%%
% 重定义字体命令
% 注意win2000,没有 simsun, 最好到网上找一个
% 一些字体是office2000 带的
%%%%%%%%%%%%%%%%%%%%%%%%%%%%%%%%%%%%%%%%%%%%%%%%%%%%%%%%%%%
%\newcommand{\song}{\CJKfamily{song}}    % 宋体   (Windows自带simsun.ttf)
%\newcommand{\fs}{\CJKfamily{fs}}        % 仿宋体 (Windows自带simfs.ttf)
%\newcommand{\kai}{\CJKfamily{kai}}      % 楷体   (Windows自带simkai.ttf)
%\newcommand{\hei}{\CJKfamily{hei}}      % 黑体   (Windows自带simhei.ttf)
%\newcommand{\li}{\CJKfamily{li}}        % 隶书   (Windows自带simli.ttf)

%\newcommand{\song}{\fontspec{SimSun}}    % 宋体   (Windows自带simsun.ttf) %[scale=0.9]
%\newcommand{\fs}{\fontspec{FangSong_GB2312}}        % 仿宋体 (Windows自带simfs.ttf) [scale=0.9]
%\newcommand{\kai}{\fontspec{KaiTi_GB2312}}      % 楷体   (Windows自带simkai.ttf)
%\newcommand{\hei}{\fontspec{SimHei}}      % 黑体   (Windows自带simhei.ttf)
%\newcommand{\li}{\fontspec{SimLi}}        % 隶书   (Windows自带simli.ttf)


\newcommand{\song}{\newfontfamily\zhfont{SimSun}}    % 宋体   (Windows自带simsun.ttf) %[scale=0.9] %SimSun
\newcommand{\fs}{\newfontfamily\zhfont{FangSong_GB2312}}        % 仿宋体 (Windows自带simfs.ttf) [scale=0.9]
\newcommand{\kai}{\newfontfamily\zhfont{KaiTi_GB2312}}      % 楷体   (Windows自带simkai.ttf)
\newcommand{\hei}{\newfontfamily\zhfont{SimHei}}      % 黑体   (Windows自带simhei.ttf)
\newcommand{\li}{\newfontfamily\zhfont{SimLi}}        % 隶书   (Windows自带simli.ttf)
\newcommand{\sectionfont}{\setromanfont{SimHei}\hei}

\setromanfont{Times New Roman}   											% 正文中的英文 采用 Times New Roman 字体;


%\iffalse
\makeatletter
\g@addto@macro\zhs@savefont{%
\long\edef\zhs@tmpmacro{\f@family}%
\def\zhs@curr@fam{0}%
\ifx\zhs@tmpmacro\sfdefault
\def\zhs@curr@fam{1}%
\else\ifx\zhs@tmpmacro\ttdefault
\def\zhs@curr@fam{2}%
\fi\fi
\edef\zhs@tmpmacro{\f@family}%
\ifx\zhs@tmpmacro\sfdefault
\def\zhs@curr@fam{1}%
\else\ifx\zhs@tmpmacro\ttdefault
\def\zhs@curr@fam{2}%
\fi\fi
}
\newfontfamily\zhrmfont[BoldFont=SimHei,
ItalicFont=KaiTi_GB2312]{SimSun} %
\newfontfamily\zhsffont{SimHei}
\newfontfamily\zhttfont[BoldFont=SimHei]{KaiTi_GB2312}
\def\zhfont{\ifcase\zhs@curr@fam\zhrmfont\or\zhsffont
\or\zhttfont\else\zhrmfont\fi}
\makeatother
%\fi

%%%%%%%%%%%%%%%%%%%%%%%%%%%%%%%%%%%%%%%%%%%%%%%%%%%%%%%%%%%
% 重定义字号命令
%%%%%%%%%%%%%%%%%%%%%%%%%%%%%%%%%%%%%%%%%%%%%%%%%%%%%%%%%%%
\newcommand{\yihao}{\fontsize{26pt}{26pt}\selectfont}       % 一号, 1.倍行距
\newcommand{\xiaoyi}{\fontsize{24pt}{24pt}\selectfont}       % 小一, 1.倍行距
\newcommand{\erhao}{\fontsize{22pt}{22pt}\selectfont}       % 二号, 1.倍行距
\newcommand{\xiaoer}{\fontsize{18pt}{18pt}\selectfont}      % 小二, 单倍行距
\newcommand{\sanhao}{\fontsize{16pt}{16pt}\selectfont}      % 三号, 1.倍行距
\newcommand{\xiaosan}{\fontsize{15pt}{15pt}\selectfont}     % 小三, 1.倍行距
\newcommand{\sihao}{\fontsize{14pt}{14pt}\selectfont}       % 四号, 1.0倍行距
\newcommand{\banxiaosi}{\fontsize{13pt}{13pt}\selectfont}   % 半小四, 1.0倍行距
\newcommand{\xiaosi}{\fontsize{12pt}{12pt}\selectfont}      % 小四, 1.倍行距
\newcommand{\dawuhao}{\fontsize{11.5pt}{11.5pt}\selectfont} % 大五号, 单倍行距
\newcommand{\wuhao}{\fontsize{10.5pt}{10.5pt}\selectfont}   % 五号, 单倍行距
\newcommand{\xiaowu}{\fontsize{9.5pt}{9.5pt}\selectfont}    % 五号, 单倍行距
\newcommand{\banbanxiaosi}{\fontsize{12pt}{12pt}\selectfont}% 半半小四, 1.0倍行距


%避免宏包 hyperref 和 arydshln 不兼容带来的目录链接失效的问题。
%\def\temp{\relax}
%\let\temp\addcontentsline
%\gdef\addcontentsline{\phantomsection\temp}
%\newcommand*{\subfigencaptionlist}{} % 子图形加入目录时用

\makeatletter
\gdef\hitfor{\@for}
\gdef\hitempty{}
\gdef\hittwo{\tw@}
\makeatother

\newcommand{\mr}[1]{\mathrm{#1}} %定义新命令,用\mr来代替\mathrm
\renewcommand \refname { 主要参考文献 }
\renewcommand \bibname { 主要参考文献 }

%定义图表章节双标题命令
\newcommand{\figenname}{Fig.}
\newcommand{\listfigenname}{List of Figures}
\newfloatlist[chapter]{figen}{fen}{\listfigenname}{\figenname}
\newfixedcaption{\figencaption}{figen}
\renewcommand{\thefigen}{\arabic{figure}}
\makeatletter
\renewcommand{\@cftmakefentitle}{\chapter*{\listfigenname\@mkboth{\bfseries\listfigenname}{\bfseries\listfigenname}}}
\makeatother

\newcommand{\FigureBiCaption}[2] %双语标题
{\renewcommand{\figurename}{图}
\vspace{-0.9ex}\caption{\protect\setlength{\baselineskip}{1.5em}#1} %\protect\setlength{\baselinestretch}{1.3}\selectfont
\vspace{-1.4ex}
\figencaption{\protect\setlength{\baselineskip}{1.5em}#2}%
%%其子图形加入目录
\makeatletter
\def\hittemp{}
 \hitfor \hittemp:=\subfigencaptionlist \do {%
        \ifx \hitempty\hittemp\relax \else
          \addcontentsline{fen}{subfigen}{\protect\numberline\hittemp}
        \fi}
 \gdef\subfigencaptionlist{}
\makeatother
}

\newcommand{\FigCaption}[1] %单语标题
{\renewcommand{\figurename}{图}
\vspace{-0.9ex}\caption{\protect\setlength{\baselineskip}{1.5em}#1} %\protect\setlength{\baselinestretch}{1.3}\selectfont
}

\setcounter{fendepth}{2} %英文图形目录的深度 1(只有一级目录) 2(有两级目录)
\setcounter{lofdepth}{2} %中文图形目录的深度 1(只有一级目录) 2(有两级目录)
\makeatletter
\renewcommand*{\l@subfigure}{\@dottedxxxline{\ext@subfigure}{2}{3.8em}{1.5em}} %中文图形目录 subfigure
\gdef\l@subfigen{\@dottedtocline{0}{3.8em}{1.5em}}%英文图形目录 latex
\newif\ifsubfigtoc
\ifnum \tw@ > \@nameuse{c@fendepth} \subfigtocfalse \else \subfigtoctrue \fi
\makeatother
\newbox\tempbox
\newcommand{\SubfigEnCaption}[1]
{\makeatletter
 \ifsubfigtoc
    %加入目录这个动作,一定要在 父图 之后,所在先暂存在 subfigencaptionlist
    \xdef\subfigencaptionlist{\subfigencaptionlist,%
        {{\thesubfigure}\protect\ignorespaces{#1}}}
\else
    \relax
\fi
\makeatother
%产生caption
\vspace{-1.8ex}
\sbox{\tempbox}{\thesubfigure\hskip\subfiglabelskip #1}%
\ifthenelse{\lengthtest{\wd\tempbox > \linewidth}}%
{\\\parbox[t]{\linewidth}{\flushleft\noindent\song\rmfamily\wuhao\selectfont\thesubfigure\hskip\subfiglabelskip \centering#1\hangafter=1\hangindent=15pt}}%[scale=0.9]
{\\[2ex]\centerline{\song\rmfamily\wuhao\selectfont\thesubfigure\hskip\subfiglabelskip #1}} %[scale=0.9]
}

\newcommand{\tblenname}{Table} %define tbl instead of table
\newcommand{\listtblenname}{List of Tables}
\newfloatlist[chapter]{tblen}{ten}{\listtblenname}{\tblenname}
\newfixedcaption{\tblencaption}{tblen}
\renewcommand{\thetblen}{\arabic{table}}% 将tblen换成table,因为table和tablen编号一致,而tablen在\longbitoccaption定义中无效。
\makeatletter
\renewcommand{\@cftmaketentitle}{\chapter*{\listtblenname\@mkboth{\bfseries\listtblenname}{\bfseries\listtblenname}}}
\makeatother

\newcommand{\TableBiCaption}[2]%双语标题
{
\renewcommand{\tablename}{表}
\caption{\protect\setlength{\baselineskip}{1.5em}#1}
\vspace{-2ex}
\tblencaption{\protect\setlength{\baselineskip}{1.5em}#2}
\vspace{1ex}
}

\newcommand{\TabCaption}[1] %单语标题
{ \renewcommand{\tablename}{表}
  \caption{\protect\setlength{\baselineskip}{1.5em}#1}%
  \vspace{1.2ex}
}

%%%% 长表格的caption在中英文表格目录中正常显示
\makeatletter
\def\@cont@LT@LTBiToeCaption#1[#2]#3{%
  \LT@makecaption#1\fnum@table{#3}%
  \def\@tempa{#2}%
  \ifx\@tempa\@empty\else
    {\let\\\space
      %\phantomsection
      \addcontentsline{ten}{tblen}{\protect\numberline{\thetable}{#2}}}%%\addcontentsline{lot}{table}{\protect\numberline{}{#2}}}%
  \fi}
\def\LT@c@ption#1[#2]#3{%
  \LT@makecaption#1\fnum@table{#3}%
  \def\@tempa{#2}%
  \ifx\@tempa\@empty\else
     {\let\\\space
     %\phantomsection
     \addcontentsline{lot}{table}{\protect\numberline{\thetable}{#2}}}%
  \fi}
\let\@cont@oldLT@c@ption\LT@c@ption
\newcommand*{\LTBiTocCaption}[5]{% 双语标题
  \@if@contemptyarg{#1}{\caption{#2}}{\caption[#1]{#2}}%
  \global\let\@cont@oldtablename\tablename
  \gdef\tablename{Table} %#3
  \global\let\LT@c@ption\@cont@LT@LTBiToeCaption
  \\
  \@if@contemptyarg{#4}{\caption{#5}}{\caption[#4]{#5}}%
  \global\let\tablename\@cont@oldtablename
  \global\let\LT@c@ption\@cont@oldLT@c@ption}
\makeatother

\makeatletter
\newcommand*{\LTCaption}[1]{ % 单语标题
 \caption{#1}
}
\makeatother

\renewcommand{\cfttblendotsep}{1} %自定义图表目录中的点间距大小
\renewcommand{\cftfigendotsep}{1}

%%%---公式中符号描述----start----
%\begin{formulasymb}{式中}{-3pt}%-3pt,-20pt调与上方的间距。
%  \fdesfirst{第一标签}{控制控制控制控制控制}
%  \fdes{其他标签}{控制控制控制控制控制}
%\end{formulasymb}
\newenvironment{formulades}[1]%
{\noindent\begin{list}{}{%
\setlength\topsep{0pt}
\settowidth{\labelwidth}{#1}
\setlength{\labelsep}{1mm}
\setlength{\leftmargin}{\labelwidth+\labelsep}
}}{\end{list}}
\newenvironment{formulasymb}[2]%-\!-\!-\!-
{\vspace*{#2}\newcommand{\fdesfirst}[2]%
{\begin{formulades}{#1\hspace*{26pt}##1~\cdash}\item[#1\hspace*{26pt}##1~\cdash]{##2}\end{formulades}\vspace*{-21pt}}%自己调距
\newcommand{\fdes}[2]{\begin{formulades}{#1\hspace{26pt}##1~\cdash}\item[##1~\cdash]{##2}\end{formulades}\vspace*{-21pt}}}%自己调距
{\vspace{21pt}\relax}%21pt调距
%%----公式中符号描述----end-----

%% ---- 左对齐的公式 start-----   \begin{flualign} a=c \end{flualign}
\newenvironment{flualign}{%
    \@fleqntrue
    \@mathmargin = -1sp
    \@mathmargin\leftmargini minus\leftmargini
    \let\mathindent=\@mathmargin
  \start@align\@ne\st@rredfalse\m@ne
}{%
  \math@cr \black@\totwidth@
  \egroup
  \ifingather@
    \restorealignstate@
    \egroup
    \nonumber
    \ifnum0=`{\fi\iffalse}\fi
  \else
    $$%
  \fi
  \ignorespacesafterend
  \@fleqnfalse
}
%%  ---- 左对齐的公式  end----

% 不再用BiChapter、 BiSection, 直接 Chapter Section .

\renewcommand{\thefigure}{\arabic{figure}}%使图编号为 7-1 的格式 %\protect{~}
%\makeatletter
%\renewcommand\fnum@figure{\figurename\nobreakspace\thefigure\protect{~~~~~~~~~}} %
%\makeatother

\renewcommand{\thesubfigure}{\alph{subfigure})}%使子图编号为 a)的格式
\makeatletter
\renewcommand{\p@subfigure}{\thefigure(} %%使子图引用为 7-1(a) 的格式
\makeatother
%\renewcommand{\thesubfigure}{\alph{subfigure}}
%\makeatletter
%\renewcommand{\p@subfigure}{\thefigure} %%使子图引用为 7-1a 的格式
%\renewcommand{\@thesubfigure}{\thesubfigure)\hskip\subfiglabelskip}%使子图编号为 a)的格式
%\makeatother

\renewcommand{\thetable}{\arabic{table}}%%使表编号为 7-1 的格式
\renewcommand{\theequation}{\arabic{equation}}%%使公式编号为 7-1 的格式

%定义 学科 学位
\def \xuekeEngineering {Engineering}
\def \xuekeScience {Science}
\def \xuekeManagement {Management}
\def \xuekeArts {Arts}

\ifx \xueke \xuekeEngineering
\newcommand{\cxueke}{工学}
\newcommand{\exueke}{Engineering}
\fi

\ifx \xueke \xuekeScience
\newcommand{\cxueke}{理学}
\newcommand{\exueke}{Science}
\fi

\ifx \xueke \xuekeManagement
\newcommand{\cxueke}{管理学}
\newcommand{\exueke}{Management}
\fi

\ifx \xueke \xuekeArts
\newcommand{\cxueke}{文学}
\newcommand{\exueke}{Arts}
\fi

\newcommand{\cdash}{\mbox{—\!\!\!\!—\!\!\!\!—}}%输入中文破折号的命令
\newcommand{\dif}{\mathrm{d}}%在数学模式中输入微分dx
