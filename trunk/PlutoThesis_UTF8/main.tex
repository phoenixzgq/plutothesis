\def\usewhat{dvipspdf}    % 你喜欢哪种编译方式,pdflatex dvipdfmx dvipspdf xelatex yap

%input "reference\reference.bib" %for winedt users
\def\version{1.8.2.20080601}         % 该变量仅用于模板文件的版本号控制

\def \xuewei {Doctor}   % 定义学位 博士
%\def \xuewei {Master}    % 硕士

\def\oneortwoside{twoside} %定义单双面打印,只对硕士学位论文有效�?
%\def\oneortwoside{oneside} % 硕士单面打印

\def\xueke{Engineering} % 定义学科 工学
%\def\xueke{Science}      % 理学
%\def\xueke{Management}   % 管理�?
%\def\xueke{Arts}         % 艺术�?

%定义xelatex的中间临时变量,若\usewhat为xelatex时,后面执行xelatx的相关选项
\def\atempxetex{xelatex} %这一项无需改动

\input{setup/type.tex}    % 硕博类型

%下面的book选项中可以使�?draft 选项,使插入的图形只显示外框,以加快预览速度�?
\documentclass[12pt,a4paper,openany,\oneortwoside]{book}

%%%%%%%%%%%%%%%%%%%%%%%%%%%%%%%%%%%%%%%%%%%%%%%%%%%%%%%%%%%%%%%%%%%%%%%%%
%
%   LaTeX File for the Dissertation of Harbin Institute of Technology
%   LaTeX + CJK     ��������ҵ��ѧѧλ����ģ��
%   Based on Wang Lei's Template for Tsinghua University
%   Version: \beta
%   Last Update: 2004-04-28
%
%%%%%%%%%%%%%%%%%%%%%%%%%%%%%%%%%%%%%%%%%%%%%%%%%%%%%%%%%%%%%%%%%%%%%%%%%
%    by  Xiangqian Wu (HIT BBS: UFO, SMTH BBS: SuperUFO)
%                  csxqwu@hotmail.com
%%%%%%%%%%%%%%%%%%%%%%%%%%%%%%%%%%%%%%%%%%%%%%%%%%%%%%%%%%%%%%%%%%%%%%%%%
%%%%%%%%%%%%%%%%%%%%%%%%%%%%%%%%%%%%%%%%%%%%%%%%%%%%%%%%%%%%%%%%%%%%%%%%%
%
%   LaTeX File for Doctor (Master) Thesis of Tsinghua University
%   LaTeX + CJK     �廪��ѧ��ʿ��˶ʿ������ģ��
%   Based on Wang Tianshu's Template for XJTU
%   Version: 1.00
%   Last Update: 2003-09-12
%
%%%%%%%%%%%%%%%%%%%%%%%%%%%%%%%%%%%%%%%%%%%%%%%%%%%%%%%%%%%%%%%%%%%%%%%%%
%   Copyright 2002-2003  by  Lei Wang (BaconChina)       (bcpub@sina.com)
%%%%%%%%%%%%%%%%%%%%%%%%%%%%%%%%%%%%%%%%%%%%%%%%%%%%%%%%%%%%%%%%%%%%%%%%%

%%%%%%%%%%%%%%%%%%%%%%%%%%%%%%%%%%%%%%%%%%%%%%%%%%%%%%%%%%%%%%%%%%%%%%%%%
%
%   LaTeX File for phd thesis of xi'an Jiao Tong University
%
%%%%%%%%%%%%%%%%%%%%%%%%%%%%%%%%%%%%%%%%%%%%%%%%%%%%%%%%%%%%%%%%%%%%%%%%%
%   Copyright 2002  by  Wang Tianshu    (tswang@asia.com)
%%%%%%%%%%%%%%%%%%%%%%%%%%%%%%%%%%%%%%%%%%%%%%%%%%%%%%%%%%%%%%%%%%%%%%%%%

%%%%%%%%%%%%%%%%%%%%%%%%%%%%%%%%%%%%%%%%%%%%%%%%%%%%%%%%%%%
%
% Latex ������ͨ��ѧ��ʿ���ĵ�ģ��.
%
% ����ʹ��miktex2.1���װ�����ģ��
%
%%%%%%%%%%%%%%%%%%%%%%%%%%%%%%%%%%%%%%%%%%%%%%%%%%%%%%%%%%

%%%%%%%%%%%%%%%%%%%%%%%%%%%%%%%%%%%%%%%%%%%%%%%%%%%%%%%%%%%
%
% ���õĺ������Ӧ�Ķ���
%
%%%%%%%%%%%%%%%%%%%%%%%%%%%%%%%%%%%%%%%%%%%%%%%%%%%%%%%%%%%


% ͼ��֧�ֺ�� Ϊ��ʹ��pdftex ��Ҫ����Ӧ�ж�

%����һ�����ж�����
\newif\ifpdf
\ifx\pdfoutput\undefined
   \pdffalse
 \else
   \pdfoutput=1
   \pdftrue
 \fi



%%%%%%%%%%%%%%%%%%%   Modified By UFO   Begin  %%%%%%%%%%%%%%%%%%%%%%%
%%%%%%%��ɫ���ú���ǩ_UFO%%%%%%
\ifpdf
\usepackage[pdftex]{graphicx}

% ���ɴ���ǩ��pdf
\else
\usepackage[dvips]{graphicx}
\fi

%\usepackage[dvipdfm,
%            CJKbookmarks=true,
%            bookmarksnumbered=true,
%            bookmarksopen=true,
%            colorlinks=true,
%            pdfborder=001,
%            citecolor=blue,
%            linkcolor=red,
%            anchorcolor=green,
%            urlcolor=blue
%            ]{hyperref}
%%%%%%%%%%%%%%%%%%%   Modified By UFO   End  %%%%%%%%%%%%%%%%%%%%%%%%%

%

% ֧�ֲ�ɫ
\usepackage{color}
% epsͼ��
\usepackage{epsfig}

% �����������
\usepackage{indentfirst}

% ������ƺ��������涨�İ���ߴ�
%%%%%%%%%%%%%%%%%%%%%%%%%%
% HIT:
%���İ�о��Сһ��ӦΪ145mm��210mm������ҳü��ҳ����Ϊ145mm��230mm��
%ҳ���ڰ�о�±���֮�¸��о��з��ã�
%%%%%%%%%%%%%%%%%%%%%%%%%%
\usepackage[%paperwidth=18.4cm, paperheight= 26cm,
            body={14.5true cm,21true cm},
            twosideshift=0 pt,
            %headheight=1.0true cm
            ]{geometry}

% ����֧�ֺ��
\usepackage{CJK}

% ��ע����
\usepackage[perpage,symbol]{footmisc}

% AMSLaTeX��� �����ų�����Ư���Ĺ�ʽ
\usepackage{amsmath}
\usepackage{amssymb}

% ��ͬ��\mathcal or \mathfrak ֮���Ӣ�Ļ�������
\usepackage{mathrsfs}

% �����໷����������� amsmath ѡ���������� AMS LaTeX �ĺ��
\usepackage[amsmath,thmmarks]{ntheorem}

% ��Ϊͼ�οɸ�������ǰҳ�Ķ��������������ܻ����
% ���������ı���ǰ��. Ҫ��ֹ�����������ʹ�� flafter
% ���
%\usepackage{flafter}

%����ͼ�ο��ƺ��
%������һ��section�ĸ���ͼ�γ�������һ��section�Ŀ�ʼ����
%�ú���ṩ������������� \FloatBarrier ���ʹ����δ��
%���ĸ���ͼ������������
\usepackage[below]{placeins}

% ͼ�Ļ����ú��
\usepackage{floatflt}

% ͼ�κͱ���Ŀ���
\usepackage{rotating}

% tex1cm�������������Ĵ�С
\usepackage{type1cm}

% ���Ʊ���ĺ��
\usepackage[sf]{titlesec}

% ����Ŀ¼�ĺ��
%\usepackage{titletoc}
\usepackage{titletoc}
% ������ѧ��ʽ�еĺ�б��ĺ��
\usepackage{bm}

%�ɽ�����������õ��ļ������
%\usepackage{endfloat}

% fancyhdr��� ҳü��ҳ�ŵ���ض���
\usepackage{fancyhdr}
\usepackage{fancyref}

%%%%%%%%%%%%%%%%%%%   Modified By UFO   Begin  %%%%%%%%%%%%%%%%%%%%%%%

% ֧�����õĺ��
%\usepackage{cite}
% ֧��������д�ĺ��_UFO
\usepackage[sort&compress,numbers]{natbib}
\usepackage{hypernat}

%%%%%%%%%%%%%%%%%%%   Modified By UFO   End  %%%%%%%%%%%%%%%%%%%%%%%%%


%����ͼ�κͱ��������ʽ
\usepackage[centerlast]{caption2}

% ���Ʊ����ͼ�εĶ��б����о�
\usepackage{setspace}

% ��ӡ��ǰҳ���ʽ�ĺ��
\usepackage{layouts}

% ʹ��Times����ĺ��
\usepackage{times}

%%%%%%%%%%%%%%%%%%%   Added By UFO   Begin  %%%%%%%%%%%%%%%%%%%%%%%
%ʹ��Multirow�����ʹ�ñ�����Ժϲ����row��
\usepackage{multirow}
%֧����ͼ
\usepackage[hang, centerlast]{subfigure}
%��ѧ���
\usepackage{amsmath}
%%%%%%%%%%%%%%%%%%%   Added By UFO   End  %%%%%%%%%%%%%%%%%%%%%%%%%
\usepackage[dvipdfm,
            CJKbookmarks=true,
            bookmarksnumbered=true,
            bookmarksopen=true,
            colorlinks=true,
            pdfborder=001,
            citecolor=black,
            linkcolor=black,
            anchorcolor=black,
            urlcolor=black
            ]{hyperref}
\AtBeginDvi{\special{pdf:tounicode GBK-EUC-UCS2}} % GBK -> Unicode
\usepackage{hyperref}
 % 引用的宏�?
%\usepackage[slantfont,boldfont,CJKaddspaces]{xeCJK} %
%\CJKlanguage{zh-cn}

% 论文包含的内�?
\includeonly{
                body/Introduction,
                body/Tricks,
                body/UpdateLog,
                body/ToTemplateMaintainers,
                body/copyright,
                body/conclusion,
                appendix/appA,
                appendix/publications,
                appendix/Authorization,
                appendix/acknowledgements,
                appendix/Resume
            }
\graphicspath{{figures/}} %定义所有的eps文件�?figures 子目录下

\begin{document}
\ifx\atempxetex\usewhat\else
\begin{CJK}{UTF8}{song}
\fi

%%%%%%%%%%%%%%%%%%%%%%%%%%%%%%%%%%%%%%%%%%%%%%%%%%%%%%%%%%%
% �ض�����������
% ע��win2000,û�� simsun,����õ�������һ��
% һЩ������office2000 ����
%%%%%%%%%%%%%%%%%%%%%%%%%%%%%%%%%%%%%%%%%%%%%%%%%%%%%%%%%%%
\newcommand{\song}{\CJKfamily{song}}    % ����   (Windows�Դ�simsun.ttf)
\newcommand{\fs}{\CJKfamily{fs}}        % ������ (Windows�Դ�simfs.ttf)
\newcommand{\kai}{\CJKfamily{kai}}      % ����   (Windows�Դ�simkai.ttf)
\newcommand{\hei}{\CJKfamily{hei}}      % ����   (Windows�Դ�simhei.ttf)
\newcommand{\li}{\CJKfamily{li}}        % ����   (Windows�Դ�simli.ttf)

%%%%%%%%%%%%%%%%%%%%%%%%%%%%%%%%%%%%%%%%%%%%%%%%%%%%%%%%%%%
% �ض����ֺ�����
%%%%%%%%%%%%%%%%%%%%%%%%%%%%%%%%%%%%%%%%%%%%%%%%%%%%%%%%%%%
\newcommand{\yihao}{\fontsize{26pt}{26pt}\selectfont}       % һ��, 1.���о�
\newcommand{\erhao}{\fontsize{22pt}{22pt}\selectfont}       % ����, 1.���о�
\newcommand{\xiaoer}{\fontsize{18pt}{18pt}\selectfont}      % С��, �����о�
\newcommand{\sanhao}{\fontsize{16pt}{16pt}\selectfont}      % ����, 1.���о�
\newcommand{\xiaosan}{\fontsize{15pt}{15pt}\selectfont}     % С��, 1.���о�
\newcommand{\sihao}{\fontsize{14pt}{14pt}\selectfont}       % �ĺ�, 1.0���о�
\newcommand{\banxiaosi}{\fontsize{13pt}{13pt}\selectfont}   % ��С��, 1.0���о�
\newcommand{\xiaosi}{\fontsize{12pt}{12pt}\selectfont}      % С��, 1.���о�
\newcommand{\dawuhao}{\fontsize{11.5pt}{11.5pt}\selectfont} % �����, �����о�
\newcommand{\wuhao}{\fontsize{10.5pt}{10.5pt}\selectfont}   % ���, �����о�
\newcommand{\xiaowu}{\fontsize{9.5pt}{9.5pt}\selectfont}    % ���, �����о�
\newcommand{\banbanxiaosi}{\fontsize{12pt}{12pt}\selectfont}% ���С��, 1.0���о�

%������ hyperref �� arydshln �����ݴ�����Ŀ¼����ʧЧ�����⡣
\def\temp{\relax}
\let\temp\addcontentsline
\gdef\addcontentsline{\phantomsection\temp}
\newcommand*{\subfigencaptionlist}{} % ��ͼ�μ���Ŀ¼ʱ��

\makeatletter
\gdef\hitfor{\@for}
\gdef\hitempty{}
\gdef\hittwo{\tw@}
\makeatother

\newcommand{\mr}[1]{\mathrm{#1}} %�����������\mr������\mathrm
\renewcommand\refname{��Ҫ�ο�����}
\renewcommand\bibname{��Ҫ�ο�����}

%����ͼ���½�˫��������
\newcommand{\figenname}{Fig.}
\newcommand{\listfigenname}{List of Figures}
\newfloatlist[chapter]{figen}{fen}{\listfigenname}{\figenname}
\newfixedcaption{\figencaption}{figen}
\renewcommand{\thefigen}{\arabic{figure}}
\makeatletter
\renewcommand{\@cftmakefentitle}{\chapter*{\listfigenname\@mkboth{\bfseries\listfigenname}{\bfseries\listfigenname}}}
\makeatother

\newcommand{\FigureBiCaption}[2] %˫�����
{\renewcommand{\figurename}{ͼ}
\vspace{-0.9ex}\caption{\protect\setlength{\baselineskip}{1.5em}#1} %\protect\setlength{\baselinestretch}{1.3}\selectfont
\vspace{-1.4ex}
\figencaption{\protect\setlength{\baselineskip}{1.5em}#2}%
%%����ͼ�μ���Ŀ¼
\makeatletter
\def\hittemp{}
 \hitfor \hittemp:=\subfigencaptionlist \do {%
        \ifx \hitempty\hittemp\relax \else
          \addcontentsline{fen}{subfigen}{\protect\numberline\hittemp}
        \fi}
 \gdef\subfigencaptionlist{}
\makeatother
}

\newcommand{\FigCaption}[1] %�������
{\renewcommand{\figurename}{ͼ}
\vspace{-0.9ex}\caption{\protect\setlength{\baselineskip}{1.5em}#1} %\protect\setlength{\baselinestretch}{1.3}\selectfont
}

\setcounter{fendepth}{2} %Ӣ��ͼ��Ŀ¼����� 1(ֻ��һ��Ŀ¼) 2(������Ŀ¼)
\setcounter{lofdepth}{2} %����ͼ��Ŀ¼����� 1(ֻ��һ��Ŀ¼) 2(������Ŀ¼)
\makeatletter
\renewcommand*{\l@subfigure}{\@dottedxxxline{\ext@subfigure}{2}{3.8em}{1.5em}} %����ͼ��Ŀ¼ subfigure
\gdef\l@subfigen{\@dottedtocline{0}{3.8em}{1.5em}}%Ӣ��ͼ��Ŀ¼ latex
\newif\ifsubfigtoc
\ifnum \tw@ > \@nameuse{c@fendepth} \subfigtocfalse \else \subfigtoctrue \fi
\makeatother
\newbox\tempbox
\newcommand{\SubfigEnCaption}[1]
{\makeatletter
 \ifsubfigtoc
    %����Ŀ¼���������һ��Ҫ�� ��ͼ ֮��,�������ݴ��� subfigencaptionlist
    \xdef\subfigencaptionlist{\subfigencaptionlist,%
        {{\thesubfigure}\protect\ignorespaces{#1}}}
\else
    \relax
\fi
\makeatother
%����caption
\vspace{-1.8ex}
\sbox{\tempbox}{\thesubfigure\hskip\subfiglabelskip #1}%
\ifthenelse{\lengthtest{\wd\tempbox > \linewidth}}%
{\\\parbox[t]{\linewidth}{\flushleft\noindent\CJKfamily{song}\rmfamily\xiaowu\selectfont\thesubfigure\hskip\subfiglabelskip #1\hangafter=1\hangindent=15pt}}%
{\\[2ex]\centerline{\CJKfamily{song}\rmfamily\xiaowu\selectfont\thesubfigure\hskip\subfiglabelskip #1}}
}

\newcommand{\tblenname}{Table} %define tbl instead of table
\newcommand{\listtblenname}{List of Tables}
\newfloatlist[chapter]{tblen}{ten}{\listtblenname}{\tblenname}
\newfixedcaption{\tblencaption}{tblen}
\renewcommand{\thetblen}{\arabic{table}}% ��tblen����table����Ϊtable��tablen���һ�£���tablen��\longbitoccaption��������Ч��
\makeatletter
\renewcommand{\@cftmaketentitle}{\chapter*{\listtblenname\@mkboth{\bfseries\listtblenname}{\bfseries\listtblenname}}}
\makeatother

\newcommand{\TableBiCaption}[2]%˫�����
{
\renewcommand{\tablename}{��}
\caption{\protect\setlength{\baselineskip}{1.5em}#1}
\vspace{-2ex}
\tblencaption{\protect\setlength{\baselineskip}{1.5em}#2}
\vspace{1ex}
}

\newcommand{\TabCaption}[1] %�������
{ \renewcommand{\tablename}{��}
  \caption{\protect\setlength{\baselineskip}{1.5em}#1}%
  \vspace{1.2ex}
}

%%%% �������caption����Ӣ�ı���Ŀ¼��������ʾ
\makeatletter
\def\@cont@LT@LTBiToeCaption#1[#2]#3{%
  \LT@makecaption#1\fnum@table{#3}%
  \def\@tempa{#2}%
  \ifx\@tempa\@empty\else
    {\let\\\space
      %\phantomsection
      \addcontentsline{ten}{tblen}{\protect\numberline{\thetable}{#2}}}%%\addcontentsline{lot}{table}{\protect\numberline{}{#2}}}%
  \fi}
\def\LT@c@ption#1[#2]#3{%
  \LT@makecaption#1\fnum@table{#3}%
  \def\@tempa{#2}%
  \ifx\@tempa\@empty\else
     {\let\\\space
     %\phantomsection
     \addcontentsline{lot}{table}{\protect\numberline{\thetable}{#2}}}%
  \fi}
\let\@cont@oldLT@c@ption\LT@c@ption
\newcommand*{\LTBiTocCaption}[5]{% ˫�����
  \@if@contemptyarg{#1}{\caption{#2}}{\caption[#1]{#2}}%
  \global\let\@cont@oldtablename\tablename
  \gdef\tablename{Table} %#3
  \global\let\LT@c@ption\@cont@LT@LTBiToeCaption
  \\
  \@if@contemptyarg{#4}{\caption{#5}}{\caption[#4]{#5}}%
  \global\let\tablename\@cont@oldtablename
  \global\let\LT@c@ption\@cont@oldLT@c@ption}
\makeatother

\makeatletter
\newcommand*{\LTCaption}[1]{ % �������
 \caption{#1}
}
\makeatother

\renewcommand{\cfttblendotsep}{1} %�Զ���ͼ��Ŀ¼�еĵ����С
\renewcommand{\cftfigendotsep}{1}

%%%---��ʽ�з�������----start----
%\begin{formulasymb}{ʽ��}{-3pt}%-3pt,-20pt�����Ϸ��ļ�ࡣ
%  \fdesfirst{��һ��ǩ}{���ƿ��ƿ��ƿ��ƿ���}
%  \fdes{������ǩ}{���ƿ��ƿ��ƿ��ƿ���}
%\end{formulasymb}
\newenvironment{formulades}[1]%
{\noindent\begin{list}{}{%
\setlength\topsep{0pt}
\settowidth{\labelwidth}{#1}
\setlength{\labelsep}{1mm}
\setlength{\leftmargin}{\labelwidth+\labelsep}
}}{\end{list}}
\newenvironment{formulasymb}[2]%-\!-\!-\!-
{\vspace*{#2}\newcommand{\fdesfirst}[2]%
{\begin{formulades}{#1\hspace*{22pt}##1~\cdash}\item[#1\hspace*{22pt}##1~\cdash]{##2}\end{formulades}\vspace*{-21pt}}%�Լ�����
\newcommand{\fdes}[2]{\begin{formulades}{#1\hspace{22pt}##1~\cdash}\item[##1~\cdash]{##2}\end{formulades}\vspace*{-21pt}}}%�Լ�����
{\vspace{21pt}\relax}%21pt����
%%----��ʽ�з�������----end-----

% ������BiChapter�� BiSection, ֱ�� Chapter Section .

\renewcommand{\thefigure}{\arabic{figure}}%ʹͼ���Ϊ 7-1 �ĸ�ʽ %\protect{~}
%\makeatletter
%\renewcommand\fnum@figure{\figurename\nobreakspace\thefigure\protect{~~~~~~~~~}} %
%\makeatother

\renewcommand{\thesubfigure}{\alph{subfigure})}%ʹ��ͼ���Ϊ a)�ĸ�ʽ
\makeatletter
\renewcommand{\p@subfigure}{\thefigure(} %%ʹ��ͼ����Ϊ 7-1(a) �ĸ�ʽ
\makeatother
%\renewcommand{\thesubfigure}{\alph{subfigure}}
%\makeatletter
%\renewcommand{\p@subfigure}{\thefigure} %%ʹ��ͼ����Ϊ 7-1a �ĸ�ʽ
%\renewcommand{\@thesubfigure}{\thesubfigure)\hskip\subfiglabelskip}%ʹ��ͼ���Ϊ a)�ĸ�ʽ
%\makeatother

\renewcommand{\thetable}{\arabic{table}}%%ʹ�����Ϊ 7-1 �ĸ�ʽ
\renewcommand{\theequation}{\arabic{equation}}%%ʹ��ʽ���Ϊ 7-1 �ĸ�ʽ

\setlength\jot{2.5ex}%���ù�ʽ֮��Ĵ�ֱ���� Ĭ��value = 3pt

%���� ѧ�� ѧλ
\def \xuekeEngineering {Engineering}
\def \xuekeScience {Science}
\def \xuekeManagement {Management}
\def \xuekeArts {Arts}

\ifx \xueke \xuekeEngineering
\newcommand{\cxueke}{��ѧ}
\newcommand{\exueke}{Engineering}
\fi

\ifx \xueke \xuekeScience
\newcommand{\cxueke}{��ѧ}
\newcommand{\exueke}{Science}
\fi

\ifx \xueke \xuekeManagement
\newcommand{\cxueke}{����ѧ}
\newcommand{\exueke}{Management}
\fi

\ifx \xueke \xuekeArts
\newcommand{\cxueke}{��ѧ}
\newcommand{\exueke}{Arts}
\fi

\newcommand{\cdash}{\mbox{��\!\!\!\!��\!\!\!\!��}}%�����������ۺŵ�����
\newcommand{\dif}{\mathrm{d}}%����ѧģʽ������΢��dx
 % 文本格式定义
% ���İ�о453.55pt��700.15pt,���մ� ftp://hitgs.hit.edu.cn/���صķ��棨20061211��
% ���õİ�о�����ܿ��ⱨ�治��װ�������԰�о�Ȳ�ʿ���Ĵ�
% ҳ���ڰ�о�±���֮�¸��о��з��ã�
% ����
\setlength{\textwidth}{453.55pt}      % �ı�����
\setlength{\oddsidemargin}{0pt}   % ��� 3.25cm=0.71+2.54 % ��߾�
\setlength{\evensidemargin}{0pt}
%  ����
\setlength{\topmargin}{-32pt}       % 3.3=2.54+0.76  ҳ�� ���¶� 0.42
\setlength{\headheight}{20pt}      % 0.8  ҳü�߶�
\setlength{\headsep}{10pt}         % 0.4
\setlength{\textheight}{700.15pt}     % 21.0  �ı��߶�
\setlength{\footskip}{26pt}        %1.1  ����ҳ��
%%%%%%%%%%%%%%%%%%%%%%%%%%%%%%%%%%%%%%%%%%%%%%%%%%%%%%%%%%%
%������ʽ��ҳ��ʾ,��������Ƶ���ʽ����һҳ�ڣ�
%һҳ��ʾ���·ŵ��ڶ�ҳ�����ºܴ�Ŀհ׿ռ䣬�ܲ��ÿ�
\allowdisplaybreaks[4]

%%%%%%%%%%%%%%%%%%%%%%%%%%%%%%%%%%%%%%%%%%%%%%%%%%%%%%%%%%%
%������������ʹ���������ȱʡֵ��΢����һ�㣬�Ӷ���ֹ����
%����ռ�ݹ�����ı�ҳ�棬Ҳ���Է�ֹ�ںܴ�հ׵ĸ���ҳ�Ϸ��ú�С��ͼ�Ρ�
\renewcommand{\topfraction}{0.9999999}
\renewcommand{\textfraction}{0.0000001}
\renewcommand{\floatpagefraction}{0.9999}

%%%%%%%%%%%%%%%%%%%%%%%%%%%%%%%%%%%%%%%%%%%%%%%%%%%%%%%%%%%
% �ض���һЩ������ر���
%%%%%%%%%%%%%%%%%%%%%%%%%%%%%%%%%%%%%%%%%%%%%%%%%%%%%%%%%%%
\theoremstyle{plain} \theorembodyfont{\song\rmfamily}
\theoremheaderfont{\hei\rmfamily} %\theoremseparator{:}
\newtheorem{definition}{\hei ����}[chapter]
\newtheorem{proposition}[definition]{\hei ����}
\newtheorem{lemma}[definition]{\hei ����}
\newtheorem{theorem}{\hei ����}[chapter]
\newtheorem{axiom}{\hei ����}
\newtheorem{corollary}[definition]{\hei ����}
\newtheorem{exercise}[definition]{}

%%%%%%%%%%%%%%%%%%%%%%%%%%%%%%%%%%%%%%%%%%%%%%%%%%%%%%%%%%%%%%%%%%%
%���ԭproof�����������������⣺
%  1. proof �е�item��������
%  2. proof �е����һ����ʽ�³���һ���ڷ��顣
%\theoremsymbol{$\blacksquare$}
%\newtheorem{proof}{\hei ֤��}
\newenvironment{proof}{\noindent{\hei ֤����}}{\hfill $ \square $ \vskip 4mm}
\theoremsymbol{$\square$}
\newtheorem{example}{\hei ��}

%%%%%%%%%%%%%%%%%%%%%%%%%%%%%%%%%%%%%%%%%%%%%%%%%%%%%%%%%%%
% �������Ķ������� �����İ�ʽ
%%%%%%%%%%%%%%%%%%%%%%%%%%%%%%%%%%%%%%%%%%%%%%%%%%%%%%%%%%%
%\CJKcaption{GB_aloft}
\CJKcaption{gb_452}

\newlength \CJKtwospaces

\def\CJKindent{
    \settowidth\CJKtwospaces{\CJKchar{"0A1}{"0A1}\CJKchar{"0A1}{"0A1}}%
    \parindent\CJKtwospaces
}

\CJKtilde  \CJKindent

\setlength{\parindent}{26pt} %���ڹ������ĵ�ÿ�е��־�Ӵ��ˣ���Ҫ���Ӷ�����2pt

\renewcommand\contentsname{\hei Ŀ~~~~¼}

%%%%%%�½ڱ���Ϊ����1�¡�����ʽ
\renewcommand\chaptername{}%��
%%%%%%%%%%%%%%%%%%%%%%%%%%%%%%%%%%%%%%%%%%%%%%%%%%
%��������½ڵı����Ŀ¼��ĸ�ʽ
%%%%%%%%%%%%%%%%%%%%%%%%%%%%%%%%%%%%%%%%%%%%%%%%%%
\setcounter{secnumdepth}{4} \setcounter{tocdepth}{2}

\titleformat{\chapter}[hang]{\xiaoer\bf\filcenter\hei\sf\boldmath}{\xiaoer\chaptertitlename}{18pt}{\xiaoer}
\titlespacing{\chapter}{0pt}{8pt}{16pt}

\titleformat{\section}[hang]{\hei\sf\xiaosan\boldmath}{\xiaosan\thesection}{0.5em}{}
\titlespacing{\section}{0pt}{13pt}{13pt}
\makeatletter
\renewcommand\thesection{\@arabic \c@section} % ǰ�治�� thechapter
\makeatother

\titleformat{\subsection}[hang]{\hei\sf\sihao\boldmath}{\sihao\thesubsection}{0.5em}{}
\titlespacing{\subsection}{0pt}{8pt}{7pt}

\titleformat{\subsubsection}[runin]{\hei\sf\xiaosi\boldmath}{\thesubsubsection}{0.5em}{}[\;\;]
%\titleformat{\subsubsection}[hang]{\hei\sf\xiaosi}{\xiaosi\thesubsubsection}{0.5em}{}
\titlespacing{\subsubsection}{0pt}{3pt}{2pt}   % #2�����ļ�� #3�����ļ��

%��������Ŀ¼,����û��Ŀ¼
%\dottedcontents{chapter}[3.4em]{\vspace{0.5em}\hspace{-3.4em}\hei \bf\boldmath}{0.0em}{5pt}
%\dottedcontents{section}[1.16cm]{}{1.8em}{5pt}
%\dottedcontents{subsection}[2.00cm]{}{2.7em}{5pt}
%\dottedcontents{subsubsection}[2.86cm]{}{3.4em}{5pt}

%%%%%%%%%%%%%%%%%%%%%%%%%%%%%%%%%%%%%%%%%%%%%%%%%%%%%%%
% ����ҳü��ҳ�� ʹ��fancyhdr ���
%%%%%%%%%%%%%%%%%%%%%%%%%%%%%%%%%%%%%%%%%%%%%%%%%%%%%%%%
\newcommand{\makeheadrule}{%
\makebox[-3pt][l]{\rule[.7\baselineskip]{\headwidth}{0.4pt}}
\rule[0.85\baselineskip]{\headwidth}{2.25pt}\vskip-.8\baselineskip}
\makeatletter
\renewcommand{\headrule}{%
    {\if@fancyplain\let\headrulewidth\plainheadrulewidth\fi
     \makeheadrule}}

\pagestyle{fancyplain}

%ȥ���½ڱ����е�����
%%��Ҫע����һ�У�����ҳü���ɣ�����1��1  ���ۡ���ʽ
\renewcommand{\chaptermark}[1]{\markboth{\chaptertitlename~~ \ #1}{}}
 \fancyhf{}

%��book�ļ������,\leftmark�Զ���¼����֮����,\rightmark��¼�ڱ���
%% ҳü�ֺ� ����Ҫ�� С��
%���ݵ�˫���ӡ���ò�ͬ��ҳü��
\ifoneortwoside
  \fancyhead[CO]{\CJKfamily{song}\xiaowu\rightmark}
  \fancyhead[CE]{\CJKfamily{song}\xiaowu ��������ҵ��ѧ\cxueke\cxuewei ѧλ���Ŀ��ⱨ��}% ����ҳü����ߵľ���
  \fancyfoot[C,C]{\xiaowu\thepage}
\else
  \fancyhead[CO]{\CJKfamily{song}\xiaowu ��������ҵ��ѧ\cxueke\cxuewei ѧλ���Ŀ��ⱨ��}
  \fancyhead[CE]{\CJKfamily{song}\xiaowu ��������ҵ��ѧ\cxueke\cxuewei ѧλ���Ŀ��ⱨ��}% ����ҳü����ߵľ���
  \fancyfoot[C,C]{\xiaowu\thepage}
\fi

%%%%%%%%%%%%%%%%%%%%%%%%%%%%%%%%%%%%%%%%%%%%%%%%%%%%%%%%
% �����о�Ͷ���䴹ֱ����
%%%%%%%%%%%%%%%%%%%%%%%%%%%%%%%%%%%%%%%%%%%%%%%%%%%%%%%%
\renewcommand{\CJKglue}{\hskip 0.3pt plus 0.08\baselineskip}%�Ӵ��ּ�࣬ʹÿ��33����
%\setlength{\belowcaptionskip}{10pt}   % �Ӵ����ͱ���֮��ľ���
\setlength{\parskip}{3pt plus1pt minus1pt} % ����֮�����ֱ����
\renewcommand{\baselinestretch}{1.2}% �����о�

%%%%%%%%%%%%%%%%%%%%%%%%%%%%%%%%%%%%%%%%%%%%%%%%%%%%%%%%
% �����б������Ĵ�ֱ���
%%%%%%%%%%%%%%%%%%%%%%%%%%%%%%%%%%%%%%%%%%%%%%%%%%%%%%%%
\setitemize{itemindent=38pt,leftmargin=0pt,itemsep=-0.4ex,listparindent=26pt,partopsep=0pt,parsep=0.5ex,topsep=-0.25ex}
\setenumerate{itemindent=38pt,leftmargin=0pt,itemsep=-0.4ex,listparindent=26pt,partopsep=0pt,parsep=0.5ex,topsep=-0.25ex}
\setdescription{itemindent=38pt,leftmargin=0pt,itemsep=-0.4ex,listparindent=26pt,partopsep=0pt,parsep=0.5ex,topsep=-0.25ex}

%%�����
\renewcommand\bibsection{\section*{\centerline{\refname}}\vspace{-6pt}\markboth{��������ҵ��ѧ\cxueke\cxuewei ѧλ���Ŀ��ⱨ��}{\refname}} %���У������ļ��
\renewcommand\@biblabel[1]{[#1]\hspace{0.5em}} %�ο������������ߵ�����
\newcommand{\ucite}[1]{$^{\mbox{\scriptsize \cite{#1}}}$} % ���� \ucite ����ʹ��ʾ������Ϊ�ϱ���ʽ
\newcommand{\citeup}[1]{$^{\mbox{\scriptsize \cite{#1}}}$} % for WinEdt users

%%%%%%%%%%%%%%%%%%%%%%%%%%%%%%%%%%%%%%%%%%%%%%%%%%%%%%%%%%%
% ���Ƹ���ͼ�κͱ��������ʽ %������ccaption��ȫ������caption2�Ĺ���
\captionstyle{\centering}   %��ͬ��ͼ������ʽ���ò�ͬ������
%\indentcaption{0pt}           %�μ�ccaption
\hangcaption
\captionnamefont{\CJKfamily{song}\rmfamily\wuhao\selectfont}
\captiontitlefont{\CJKfamily{song}\rmfamily\wuhao\selectfont}
\captiondelim{~} %~

%%%%%%%%%%%%%%%%%%%%%%%%%%%%%%%%%%%%%%%%%%%%%%%%%%%%%%%
% ������ͷ���Եĸ�ʽ
% �÷� \begin{Aphorism}{author}
%         aphorism
%      \end{Aphorism}
\newsavebox{\AphorismAuthor}
\newenvironment{Aphorism}[1]
{\vspace{0.5cm}\begin{sloppypar} \slshape
\sbox{\AphorismAuthor}{#1}
\begin{quote}\small\itshape }
{\\ \hspace*{\fill}------\hspace{0.2cm} \usebox{\AphorismAuthor}
\end{quote}
\end{sloppypar}\vspace{0.5cm}}

%�Զ���һ�����������ע�͵��ı��в���Ҫ�IJ��֡�
\newcommand{\comment}[1]{}

\renewcommand\contentsname{\hei Ŀ~~~~¼}
\renewcommand\listfigurename{\hei ��~~~~ͼ}
\renewcommand\listtablename{\hei ��~~~~��}

%%%%%%���±����е��������֣�һ������������Ϊ����������(1,2,3)
\renewcommand\CJKthechapter{%\CJKnumber
{\@arabic\c@chapter}}

%%%%%%��Ҫ�����о�ʹ��ҳ�����
\raggedbottom

% This is the flag for longer version
\newcommand{\longer}[2]{#1}
\newcommand{\ds}{\displaystyle}
% define graph scale
\def\gs{1.0}

%%%%%%%%%%%%%%%%%%%%%%%%%%%%%%%%%%%%%%%%%%%%%%%%%%%%%%%%%%%%%%%%%%%%%%
% �Զ�����Ŀ�б���ǩ����ʽ \begin{hitlist} �б��� \end{hitlist}
%%%%%%%%%%%%%%%%%%%%%%%%%%%%%%%%%%%%%%%%%%%%%%%%%%%%%%%%%%%%%%%%%%%%%%
\newcounter{hitctr} %�Զ����¼�����
\newenvironment{hitlist}{%%%%%�����»���
\begin{list}{{\hei (\arabic{hitctr})}} %%��ǩ��ʽ
    {
     \usecounter{hitctr}
     \setlength{\leftmargin}{0cm}     %��߽�
     \setlength{\parsep}{0ex}         %������
     \setlength{\topsep}{0pt}         %�б��������ĵĴ�ֱ����
     \setlength{\itemsep}{0ex}        %��ǩ���
     \setlength{\labelsep}{0.3em}     %��ź��б���֮��ľ���,Ĭ��0.5em
     \setlength{\itemindent}{46pt}    %��ǩ������
     \setlength{\listparindent}{27pt} %����������
    }}
{\end{list}}%%%%%

%%%%%%%%%%%%%%%%%%%%%%%%%%%%%%%%%%%%%%%%%%%%%%%%%%%%%%%%%%%%%%%%%%%%%%
% �Զ�����Ŀ�б���ǩ����ʽ \begin{publist} �б��� \end{publist}
%%%%%%%%%%%%%%%%%%%%%%%%%%%%%%%%%%%%%%%%%%%%%%%%%%%%%%%%%%%%%%%%%%%%%%
\newcounter{pubctr} %�Զ����¼�����
\newenvironment{publist}{%%%%%�����»���
\begin{list}{\arabic{pubctr}} %%��ǩ��ʽ
    {
     \usecounter{pubctr}
     \setlength{\leftmargin}{2em}     % ��߽� \leftmargin =\itemindent + \labelwidth + \labelsep
     \setlength{\itemindent}{0em}     % ���������
     \setlength{\labelwidth}{1em}     % ��ſ���
     \setlength{\labelsep}{1em}       % ��ź��б���֮��ľ���,Ĭ��0.5em
     \setlength{\rightmargin}{0em}    % �ұ߽�
     \setlength{\topsep}{0ex}         % �б��������ĵĴ�ֱ����
%     \setlength{\partopsep}{0ex}      % �б���һ���µĶ���ʱ���ӵĶ��⵽�����ĵľ���
     \setlength{\parsep}{0ex}         % ������
     \setlength{\itemsep}{0ex}        % ��ǩ���
     \setlength{\listparindent}{26pt} % ����������
    }}
{\end{list}}%%%%%

%%%%%%%%%%%%%%%%%%%%%%%%%%%%%%%%%%%%%%%%%%%%%%%%%%%%%%%%%%%%%%%%%%%%%%
% Ĭ������
\renewcommand\normalsize{%
  \@setfontsize\normalsize{12.1pt}{13pt}
  \setlength\abovedisplayskip{10pt plus 2pt minus 2pt}
  \setlength\abovedisplayshortskip{8pt plus 2pt minus 2pt}
  \setlength\belowdisplayskip{\abovedisplayskip}
  \setlength\belowdisplayshortskip{\abovedisplayshortskip}
  \setlength\jot{8pt}
  \let\@listi\@listI}
\def\defaultfont{\renewcommand{\baselinestretch}{1.37}\normalsize\selectfont}

%%%%%%%%%%%%%%%%%%%%%%%%%%%%%%%%%%%%%%%%%%%%%%%%%%%%%%%%%%%%%%%%%%%%%%
% ���桢ժҪ����Ȩ����л��ʽ����
\makeatletter
\def\caffil#1{\def\@caffil{#1}}\def\@caffil{}
\def\csubject#1{\def\@csubject{#1}}\def\@csubject{}
\def\csupervisor#1{\def\@csupervisor{#1}}\def\@csupervisor{}
\def\cassosupervisor#1{\def\@cassosupervisor{~ & {\textbf{��\hfill ��\hfill ʦ}}& \rule[-3pt]{201pt}{1.2pt}\hspace{-201pt}\centerline{#1}\\[16pt]}}\def\@cassosupervisor{}
\def\cauthor#1{\def\@cauthor{#1}}\def\@cauthor{}
\def\cbdate#1{\def\@cbdate{#1}}\def\@cbdate{} %��ѧʱ��
\def\cdate#1{\def\@cdate{#1}}\def\@cdate{}    %��������
\def\ctitle#1{\def\@ctitle{#1}}\def\@ctitle{}
\def\ctitlesec#1{\def\@ctitlesec{~ & ~ & \rule[-3pt]{201pt}{1.2pt}\hspace{-201pt}\centerline{#1}}}\def\@ctitlesec{}
%%%%%%%%%%%%%%%%%%%%%%%%%%%%%%%%%%%%%%%%%%%%%%%%%%%%%%%%%%%%%%%
% �������
\ifxueweidoctor
\def\makecover{ \normalbiao
    \setboolean{@twoside}{true}
    %%%%%%%%%%%%%����һ
    \thispagestyle{empty}
    \vspace*{42pt}
    \begin{center}
    {\kai\xiaoer \makebox[152pt][s]{\textbf{��������ҵ��ѧ}}}
    \end{center}

    \vspace{16pt}
    \begin{center}
      {\song\erhao\makebox[258pt][c]{\textbf{\xueweishort\hfill ʿ\hfillѧ\hfillλ\hfill��\hfill��\hfill��\hfill��\hfill��\hfill��}}}
    \end{center}

    \vspace{64.8pt}
    {\song\sanhao
    \def\oldtabcolsep{\tabcolsep}
    \setlength{\tabcolsep}{0pt}
    \noindent\begin{tabular}{p{69pt}p{97pt}p{201pt}lll}\setlength{\tabcolsep}{-30pt}
    ~ & {\textbf{Ժ\hfill (ϵ)}}                                   & \rule[-3pt]{201pt}{1.2pt}\hspace{-201pt}\centerline{\@caffil}\\[16pt]
    ~ & {\textbf{ѧ\hfill ��}}                                     & \rule[-3pt]{201pt}{1.2pt}\hspace{-201pt}\centerline{\@csubject}\\[16pt]
    ~ & {\textbf{��\hfill ʦ}}                                     & \rule[-3pt]{201pt}{1.2pt}\hspace{-201pt}\centerline{\@csupervisor}\\[16pt]
    \@cassosupervisor
    ~ & {\textbf{��\hfill ��\hfill ��}}                            & \rule[-3pt]{201pt}{1.2pt}\hspace{-201pt}\centerline{\@cauthor}\\[16pt]
    ~ & {\textbf{��\hfill ѧ\hfill ʱ\hfill ��}}                   & \rule[-3pt]{201pt}{1.2pt}\hspace{-201pt}\centerline{\@cbdate}\\[16pt]
    ~ & {\textbf{��\hskip-0.4pt\hfill ��\hskip-0.4pt\hfill ��\hskip-0.4pt\hfill ��\hskip-0.4pt\hfill ��\hskip-0.4pt\hfill ��}} & \rule[-3pt]{201pt}{1.2pt}\hspace{-201pt}\centerline{\@cdate}\\[16pt]
    ~ & {\textbf{��\hfill ��\hfill ��\hfill Ŀ}}                   & \rule[-3pt]{201pt}{1.2pt}\hspace{-201pt}\centerline{\@ctitle}\\[16pt]
    \@ctitlesec
    \end{tabular}
    \def\tabcolsep{\oldtabcolsep}
    \vspace{154pt}
    \begin{center}
    \textbf{�о���Ժ������}
    \end{center}
}

%%�����ڷ�
\newpage
\thispagestyle{empty}
\vspace*{14pt}
\begin{center}
  {\hei\sanhao \makebox[85pt][s]{˵\hfill ��}}
\end{center}
\vspace*{40pt}
    {\song\renewcommand\baselinestretch{1.27}
    \fontsize{10.5pt}{12.6pt}\selectfont
    \noindent һ�����ⱨ��Ӧ����������Ҫ���ݣ�
    \begin{enumerate}[leftmargin=36pt,itemindent=-2pt,itemsep=0ex,listparindent=21.8pt,partopsep=0pt,parsep=0.5ex,topsep=-0ex]
    \item ������Դ���о���Ŀ�ĺ����壻
    \item �������ڸ÷�����о���״��������������������
    \item ǰ�ڵ������о���������֤�����Ľ����
    \item ѧλ���ĵ���Ҫ�о����ݡ�ʵʩ���������������֤��
    \item ���Ľ��Ȱ��ţ�Ԥ�ڴﵽ��Ŀ�ꣻ
    \item Ϊ��ɿ����Ѿ߱����������������Э�ƻ������ѣ�
    \item Ԥ���о������п������������ѡ����⣬�Լ������;����
    \item ��Ҫ�ο����ף�Ӧ��50ƪ���ϣ������������ϲ����ڶ���֮һ��
    �ο������н������ڷ���������һ�㲻��������֮һ���ұ����н�����
    �ڷ������������ϣ���
    \end{enumerate}
    \noindent �������ⱨ������Ӧ������1.5���֡�

    \noindent �������ⱨ��ʱ��Ӧ���Ӧ�ڵ���ѧ�ڽ���ǰ��ɡ�

    \noindent �ġ������ο��ⱨ��δͨ����\hspace{-1pt}�������������ٴν��п��ⱨ�档\hspace{-1pt}�ڶ���ѧλ���Ŀ��ⱨ��
    ��δͨ��\\\hspace*{20.6pt}�ߣ���ȡ����ѧ����

    \noindent �塢���ⱨ�������\hfill ����С��Ҫ��д\hfill��\cxuewei ѧλ���Ŀ��ⱨ����������\hfill �ϱ�Ժ\hfill��ϵ��\hfill ��
    ������\\\hspace*{20.6pt}ѧ���鱸����%\hspace*{20.5pt}

    \noindent �����˱�������дʱ�������Ӹ�ҳ��
    }
    \clearpage
    \ifoneortwoside
    \else
        \setboolean{@twoside}{false}
    \fi 
\wuhaobiao }%\makecover
\fi

\ifxueweimaster
\def\makecover{ \normalbiao
    \setboolean{@twoside}{true}
    %%%%%%%%%%%%%����һ
    \thispagestyle{empty}
    \vspace*{42pt}
    \begin{center}
    {\kai\xiaoer \makebox[152pt][s]{\textbf{��������ҵ��ѧ}}}
    \end{center}

    \vspace{16pt}
    \begin{center}
      {\song\xiaoyi\makebox[258pt][c]{\textbf{\xueweishort\hfill ʿ\hfillѧ\hfillλ\hfill��\hfill��\hfill��\hfill��\hfill��\hfill��}}}
    \end{center}
    \vspace{36pt}
    \noindent{\song\xiaoer\textbf{��\ Ŀ��}\parbox[b]{400pt}{\textbf{\@ctitle}}}\\[64.8pt]
    {\song\sanhao
    \def\oldtabcolsep{\tabcolsep}
    \setlength{\tabcolsep}{0pt}
    \noindent\begin{tabular}{p{69pt}p{97pt}p{201pt}lll}\setlength{\tabcolsep}{-30pt}
    ~ & {\textbf{Ժ\hfill (ϵ)}}                                   & \rule[-3pt]{201pt}{1.2pt}\hspace{-201pt}\centerline{\@caffil}\\[16pt]
    ~ & {\textbf{ѧ\hfill ��}}                                     & \rule[-3pt]{201pt}{1.2pt}\hspace{-201pt}\centerline{\@csubject}\\[16pt]
    ~ & {\textbf{��\hfill ʦ}}                                     & \rule[-3pt]{201pt}{1.2pt}\hspace{-201pt}\centerline{\@csupervisor}\\[16pt]
    \@cassosupervisor
    ~ & {\textbf{��\hfill ��\hfill ��}}                            & \rule[-3pt]{201pt}{1.2pt}\hspace{-201pt}\centerline{\@cauthor}\\[16pt]
    ~ & {\textbf{��\hfill ��}}                   & \rule[-3pt]{201pt}{1.2pt}\hspace{-201pt}\centerline{\@cbdate}\\[16pt]
    ~ & {\textbf{��\hskip-0.4pt\hfill ��\hskip-0.4pt\hfill ��\hskip-0.4pt\hfill ��\hskip-0.4pt\hfill ��\hskip-0.4pt\hfill ��}} & \rule[-3pt]{201pt}{1.2pt}\hspace{-201pt}\centerline{\@cdate}\\[16pt]
    \end{tabular}
    \def\tabcolsep{\oldtabcolsep}
    \vfill%\vspace{154pt}
    \begin{center}
    \textbf{�о���Ժ��������}
    \end{center}
}

%%�����ڷ�
\newpage
\thispagestyle{empty}
\vspace*{44pt}
\begin{center}
  {\hei\sanhao \makebox[85pt][s]{˵\hfill ��}}
\end{center}
\vspace*{40pt}
    {\song\renewcommand\baselinestretch{1.27}
    \fontsize{10.5pt}{12.6pt}\selectfont
    \noindent һ�����ⱨ��Ӧ����������Ҫ���ݣ�
    \begin{enumerate}[leftmargin=36pt,itemindent=-2pt,itemsep=0ex,listparindent=21.8pt,partopsep=0pt,parsep=0.5ex,topsep=-0ex]
    \item ������Դ���о���Ŀ�ĺ����壻
    \item �������ڸ÷�����о���״��������
    \item ��Ҫ�о����ݣ�
    \item �о����������Ȱ��ţ�Ԥ�ڴﵽ��Ŀ�ꣻ
    \item Ϊ��ɿ����Ѿ߱�������������;��ѣ�
    \item Ԥ���о������п������������ѡ����⣬�Լ�����Ĵ�ʩ��
    \item ��Ҫ�ο����ס�
    \end{enumerate}
    \noindent �����Կ��ⱨ���Ҫ��
   \begin{enumerate}[leftmargin=36pt,itemindent=-2pt,itemsep=0ex,listparindent=21.8pt,partopsep=0pt,parsep=0.5ex,topsep=-0ex]
    \item ���ⱨ�������Ӧ��5000�����ϣ�
    \item �Ķ�����Ҫ�ο�����Ӧ��20ƪ���ϣ�������������Ӧ����������֮һ��˶ʿ�о���Ӧ��
    ��ʦ��ָ�������ز��Ľ����ڷ������С������ڿ����¡���ѧ�ƵĻ�����רҵ�ν̲�һ�㲻Ӧ
    ��Ϊ�ο����ϡ�
    \end{enumerate}
    \noindent �������ⱨ��ʱ��Ӧ��ٲ��ó�������ѧ�ڵĵ�����ĩ��

    \noindent �ġ�\hspace{-1.5pt}��˶ʿ���״ο��ⱨ��δͨ����\hspace{-1.5pt}����һ�������ٽ���һ�Ρ�\hspace{-1.5pt}���Բ�ͨ����\hspace{-1.5pt}��ֹͣ˶ʿ���Ĺ�����

    \noindent �塢�˱�������дʱ�������Ӹ�ҳ��

    \noindent �������ⱨ����к󣬴˱�ͬ˶ʿѧλ���Ŀ��ⱨ�����������ϵ��Ժ���о�����
    ���鴦���Ա�\\\hspace*{20.6pt}�о���Ժ������ѧԺ���м�顣
    }
    \clearpage
    \ifoneortwoside
    \else
        \setboolean{@twoside}{false}
    \fi 
\wuhaobiao}
\fi
\makeatother


%%% ����ֱ������� start %%%%%
\gdef\tpltable{\relax}
\let\tpltable\longtable
\makeatletter
\gdef\wuhaobiao{%�����
    \def\tabular{\wuhao\gdef\@halignto{}\@tabular}
    \def\endtabular{\endarray $\egroup \defaultfont}
    \def\longtable{\wuhao\tpltable}
    \def\endlongtable{\adl@LTlastrow \adl@org@endlongtable\defaultfont}
}
\gdef\normalbiao{%�����ֺ�
    \def\tabular{\gdef\@halignto{}\@tabular}
    \def\endtabular{\endarray $\egroup}
    \def\longtable{\tpltable}
    \def\endlongtable{\adl@LTlastrow \adl@org@endlongtable}
}
\wuhaobiao
\makeatother
%%% ����ֱ������� end %%%%%


%%%%%%%%%%%%%%%%%%%%%%%%%%%%%%%%%%%%%%%%%%%%%%%%%%%%%%%%%%%
% 正文部分
%%%%%%%%%%%%%%%%%%%%%%%%%%%%%%%%%%%%%%%%%%%%%%%%%%%%%%%%%%%
\frontmatter
\sloppy % 解决中英文混排的断行问题,会加入间距,但不会影响断行
\ctitle{��������ҵ��ѧ˶ʿ����\LaTeX{}ģ��} 
\cdegree{��ѧ˶ʿ}
\caffil{�Զ������������ϵ} %����У��������ϵ���ƣ�ͬ��ѧ����Ա�����λ��
\csubject{������ѧ�뼼��}                 %(~������ѧ����д~)
\cauthor{��~~��~~��}

%\ccosupervisor{~ij~~ij~~ij~~~~��~~��~} %(~���޸���ʦ���д���~)
\csupervisor{��~~��~~��~~~~��~~��} %��ʦ����
\cdate{2005~��~6~��}

\etitle{\LaTeX~Dissertation Template of Harbin Institute of Technology} 
\edegree{Master of Science} 
\esubject{Instrument Science and Technology}
\eauthor{WuJi Zhang}
 \esupervisor{Prof. SanFeng Zhang}
%\ecosupervisor{Professor X}
%\eassosupervisor{Professor Wen Gao}
\edate{June, 2005}

\natclassifiedindex{O1234.4}  %����ͼ������
\internatclassifiedindex{543.21}  %����ͼ������
\cabstract{ 
����ժҪ
 }

\ckeywords{\LaTeX��ѧλ���ģ�ģ��}

\eabstract{
The abstract in English.
 }

\ekeywords{\LaTeX; Dissertation Template;}

\makecover
\clearpage
 % 封面

%% 中英目录
\renewcommand{\baselinestretch}{1}
\fontsize{12pt}{12pt}\selectfont
\clearpage{\pagestyle{empty}\cleardoublepage}
\pdfbookmark[0]{目~~~~�?}{mulu}
\tableofcontents    % 中文目录
\ifxueweidoctor     % 英文目录右开
  \clearpage{\pagestyle{empty}\cleardoublepage}
\else%
  \ifoneortwoside\clearpage{\pagestyle{empty}\cleardoublepage}\fi
\fi
\renewcommand{\baselinestretch}{1.3}
\fontsize{12pt}{12pt}\selectfont
\ifxueweidoctor %硕士学位论文没有英文目录
  \tableofengcontents % 英文目录
\fi

\input{setup/figtab.tex}  %图表索引, 如果不需要图表索引,注释掉这一句即可;
% \notation  %主要符号�?
\addtocontents{toc}{\protect\vskip1\baselineskip} % 中文目录增加空行
\addtocontents{toe}{\protect\vskip1\baselineskip} % 英文目录增加空行

\ifxueweidoctor
  \clearpage{\pagestyle{empty}\cleardoublepage}   % 清除目录后面空页的页眉和页脚
\else%
  \ifoneortwoside\clearpage{\pagestyle{empty}\cleardoublepage}\fi  % 清除目录后面空页的页眉和页脚
\fi                                               %  第一�?是否右开

\mainmatter
\defaultfont % 对应于小四的标准字号�?12pt, 可以在正文中用此命令修改所需要字体的的大�?
  
\defaultfont

\BiChapter{����}{Introduction}
\label{Introduction}

\BiSection{���ⱳ��������}{The Background and Significance}
\label{Introduction:background}
\LaTeX~���ھ����Ű����ۡ��Թ�ʽ��ͼ���Ĵ�������ǿ���Լ���ƽ̨ͨ����ǿ�����ƣ�
ʹ�����ڿƼ��Ű��е�Ӧ��Խ��Խ�㷺��

\BiSection{�й�˵��}{Readme}
\BiSubsection{��������}{Environment of Software}
��ģ����\LaTeX{}+CJK�����¾����������룬����ijЩ�����������¿��ܻ�����һ
Щ�������⣬��˽���ʹ�������Ƽ�������������
\begin{hitlist}
\item WindowsNT/2000/XP+CTeX��CTeX��Ŀǰ����Ӱ������������TeX������CTeX
������װ���㣬�����˴�������õ�������������뿼��̫����������������
��ֻ��רע�����ĵĻ���CTeX�Ǹ�������ѡ��http://www.ctex.org��CTeX����
ҳ����������Ի�����µ���Ϣ������TeX�İ���(CTeX��̳)�����µ�������
\item WindowsNT/2000/XP+ChinaTeX��ChinaTeX����һ����TeX���а棬��Ҫ��������
��ʿ(hooklee)ά����ChinaTeX����iso��ʽ���ŵģ�����TeXϵͳ������һЩ�dz�
���õ����������ϡ�ChinaTeX�����˼·��CTeX������ͬ��Ŀǰ�汾�IJ���MiKTeX
Direct CD��ʽ���û��и��������ȥ����ChinaTeX��ChinaTeX����ҳ��
http://www.chinatex.org������Ҳ����ص���̳��
\item Linux+TeXlive��TeXlive��һ��������TeX���а棬֧���ڶ�IJ���ϵͳ��
����û�ж����ĵ�ֱ��֧�֣���Ҫ�����������壬�������÷������Բο���
http://learn.tsinghua.edu.cn/homepage/2001315450/tex\_frame.html��
\end{hitlist}

���������������������ԣ��������������ģ�壬���������������¿���������
������ȱ�����������ȱ�ٺ�������������Ӧ���⣬��ӭ���϶���BBS��TeX��
���ۡ�


\BiSubsection{���Ŀ¼���ļ�}{The Related Directories and Files}
��~\ref{Introduction:Tab1}��������ģ����ص�Ŀ¼���ļ���˵�������������˽���Щ�ļ�����;����������ģ��Ľṹ��ܣ�
Ȼ��򿪸����ļ����鿴ע�ͺ�һЩ�������ϸ���˽⣬Ϊʹ�ô������û�����

\begin{table}[htb]
\centering
\TableBiCaption{ģ��Ŀ¼���ļ�˵��}{Description of Directories and Files}
\label{Introduction:Tab1}
\begin{tabular}{lp{11cm}}
\hline \hline
main.tex    & ���ļ�\\
preface     & ǰ�Բ��֣��������棬����ժҪ��Ӣ��ժҪ����Ҫ���ű���\\
body        & ���IJ��֣��������ĸ��½ںͽ���\\
appendix    & ��¼���֣�������л����¼�½ں͸��˼����������������б���\\
figures     & ������в�ͼ��Ŀ¼\\
reference   & ��Ųο�����.bib�ļ���Ŀ¼\\
setup       & ��������ļ���Ŀ¼������package.tex�����Ժ������
              �úͲ������ã�format.tex��������ĸ�ʽ�����Ͷ��壬
              Define.tex��������һЩ��صĶ��壬�û�һ������Ķ�\\
readme.pdf  & һ�������õ�����������ʾ��������ģ��ʹ�ü�����˵�� \\
make.bat    & ��~Windows���Զ�ѡ��~dvips ��~dvipdfm ��~pdfLaTeX ����������Զ����������ļ��Ľű��ļ�
              �������˽���������\\
clean.bat   & ����ɾ�����б༭�ͱ���ʱ��������ʱ�ļ���pdf�ļ�����\\
chinesebst.bst & ���ɰ������IJο����׵ı�׼��ʽ�Ķ����ļ�\\
makefile    & linux�������Զ������������õ��ļ�\\
gb\_452.cap & aloft��gb.cap��4.5.2�棬���������ĸ�ʽ�йصĻ������塣
              BaconChina��ԭʼ�汾�����������޸ģ����������������汾����\\
gb\_452.cpx & ��gb\_452.cap������ȫһ�����ļ�����ͬ��\LaTeXϵͳҪ��
              ��ͬ���ļ���׺�������ļ���֤�˼�����\\
\hline \hline
\end{tabular}
\end{table}

\BiSubsection{��������ģ�����ؼ����·���}{Downloading and Updating Methods of Updated Template}
\BiSubsubsection{����ģ������}{Template Downloading Method}
��������ҵ��ѧ~ PlutoThesis ˶��ʿѧλ����ģ���ά��վ����~\url{http://gf.cs.hit.edu.cn/projects/plutothesis/} ��
�������``�ļ�'' ҳ��~\url{http://gf.cs.hit.edu.cn/frs/?group_id=91}���������ص�����������ģ�壬
���Ҳ쿴ÿ�������汾�ĸ�����־�������ģ����˽⣨��Ȼ����Ҳ�����������µ�ģ�壬Ȼ�����е�readme.pdf������ϸ����
����������µ�ģ�壬��Ϊ���µ�ģ���������ѷ��ֵ�һЩʹ�û������Ĺ淶����ȫ�Ǻϵ�~bugs����ʹ����Ҳ���ܸ�����һЩ��

����δ�����汾������ά���İ汾�������Ѿ��޸��˲���bug������ijЩbug��û�н��������������˽⣬
����ʹ�ð汾����ϵͳ~subversion �Ŀͻ��˷���~\url{https://svn.gf.cs.hit.edu.cn/svn/plutothesis/}�������������Ȥ�İ汾��
subversion ���ܼ�ʹ�ý��ܵ���վ�ܶ࣬�����ͨ��~google ������
�����Ƽ�~ \url{http://www.subversion.org.cn}����һЩ�����Ľ̳̣���̳��Ҳ��һЩ���֡�
Windows ������õ�~svn �ͻ�����~tortoisesvn������Դ��������Ϻܽ��ܣ��Ҽ��˵��������ܷ��㣻linux �±ȽϺ��õ���~repidsvn��

\BiSubsubsection{����ģ����·���}{Template Updating Method}
���ز���ѹģ���ļ��������ȳ��Ա���ģ��Դ�ļ�������û�з��ִ��󣬲ſ������Լ����ĵ�׫д������
������Ѿ�ʹ����ǰ�İ汾�����˲�������׫д��������������ǰ��ϸ�鿴����ģ���еĸ��¼�¼��
��ñ����Լ�ԭ�еĵ������ļ���Ȼ�����Ķ������ļ���
���Ƶ���Ӧ�����ص���ģ���ļ��У�������ͨ����û�з������⣬����������ĸ��¡�

�鿴�Ա�ģ���ļ����µ�һ���ȽϷ���ķ�������ʹ��ijЩ������ͬ���ļ��к��ļ��ȽϹ��ܣ�����~totalcommander��
һ��������ķ�������ʹ��~subversion �Ŀͻ��˹��߷�������ģ���~subversion �ļ��ֿ⣬
�鿴�����ٺͶԱȰ汾�ı仯�ر𷽱㡣
���˽�������д����ò���~Subversion ���а汾���ƣ��Ӷ����ñ��ݹ�����Ҳ����汾������˳���������׫д������
������������д���������õ��ij������Ҳͬʱ���а汾���ơ�����һ���ܺõ�д���ͱ��ϰ�ߡ�

\BiSubsection{СС��ʾ��}{Example of Subsubsection}
������СС�ڵ�ʾ��

\BiSubsubsection{СС��1}{Subsubsection 1}
����СС�ڣ��� СС�ڵ����ݺͱ�����ţ����Ҳ�������Ŀ¼�С�

\BiSubsubsection{СС�ڣ�}{Subsubsection 2}
����СС�ڣ���

\BiSubsubsection{СС�ڣ�}{Subsubsection 3}
����СС�ڣ���

\BiSubsubsection{СС�ڣ�}{Subsubsection 4}
����СС�ڣ���

\BiSubsubsection{СС�ڣ�}{Subsubsection 5}
����СС�ڣ���

\BiSubsubsection{СС�ڣ�}{Subsubsection 6}
����СС�ڣ���

\defaultfont

\BiChapter{ģ��ʹ���е�һЩ����}{Some Tricks of Using this Template}
\label{Tricks}

\BiSection{����}{Introduction}
\label{Tricks:Introduction}
���¼򵥽���ʹ�ñ�ģ���һЩ���ɡ�\LaTeX~�Ļ����������Ͳ����ο��������\ucite{TEXGURU99,BEZOS02,SHELL02,OOSTRUM01}��

\BiSection{��Ӣ��Ŀ¼}{Chinese and English Contents}
\label{Tricks:Contents}
���ķֱ�Ϊ�¡��ڡ�С�ں�СС�ڶ����������

\begin{verb}
\BiChapter��\BiSection��\BiSubsection��\BiSubsubsection
\end{verb}

����~4 �������ʹ�÷�����~\verb"\BiChapter{���ı���}{engish}" �� ~\verb"\BiSection{���ı���}{engish}"��ʾ��
�����½ڱ���Ƚϳ���������Ѿ�ʵ�������ĺ�Ŀ¼���Զ����еĹ��ܡ�

���⣬���¹���~\verb"\BiChapter" ��֧�����������ֶ����л��Զ����У���Ŀ¼���Զ����У�
ʹ�ø�ʽΪ~\verb"\BiChapter[�ŵ�Ŀ¼�е�]{������\\�ֶ�����}{english title}"��
���ڽڼ�С�ڣ����Է���~.../setup/definition.tex ���µ���ʽ�Լ��������Ƶ����á���������������ٳ��֣���������û��
�ٸ��¶��塣�������Ҫ��JUST DO IT YOURSELF. :-)

���ڸ�¼��û���±�ŵ��£�����۵ȣ�Ҳ������һ����Ӧ������\verb"\BiAppendixChapter"��

���ڸ�¼�����±�ŵ��£�������һ����Ӧ������\verb"\BiAppChapter"��

����Щ�����о�����������������һ��Ϊ������Ŀ���ڶ���ΪӢ����Ŀ��

\BiSection{�����}{List Environment}

��ģ�涨����hitlist��publist�б�������������������������enumerate������itemize������

ʹ�÷����뿴���ӣ�
\begin{verbatim}
\begin{hitlist}
\item hitlist���Ϲ�������ģ��Ҫ��
\item publist�������ڷ������µȵط���ʹ�ã����������ò�����
\end{hitlist}
\end{verbatim}

��������γɵ�Ч�����£�
\begin{hitlist}
\item hitlist���Ϲ�������ģ��Ҫ��
\item publist�������ڷ������µȵط���ʹ�ã����������ò�����
\end{hitlist}


\BiSection{�����}{Reference}
ģ����ʹ�õ����϶�������~jdg~�ṩ��~chinesebst.bst���ο���������ȫ�������bib�ļ���д����
����~reference\textbackslash reference.bib��Ҳ��ʹ��EndNote��NoteExpress֮������׹��������Զ����ɡ�

����\ucite{zhang2002}��һ�����Ķ�����ߵ����ӡ�����\ucite{zarchan84}��һ���������ĵ����ӡ�

\BiSection{��ӡ}{Print}
\label{Tricks:Print}

ԭUFOģ��IJ�ɫ�������ִ�ӡ������Ⱥ�ɫ����Ҫ�������ҷ��顣
���ڴ�ӡʱ����ʹѡ��``���������ִ�ӡ�ɺ�ɫ''ѡ�
��ӡ����~$\Rightarrow$~����~$\Rightarrow$~��ϸ����~$\Rightarrow$~ѡ��
``���������ִ�ӡ�ɺ�ɫ''����Ȼ���ܽ��������⣬�����Ǵ�ӡϵͳ�������ַ�����ͼ������ӡ�ġ�

Ϊ��ʹ�ô�ӡ����������ۣ���ģ���Ŀ¼�͹�ʽͼ�������ã��ο����׵����þ��޸�Ϊ��ɫ��
�޸ķ�ʽΪ���� ../setup/package.tex �е�~ colorlinks ����Ϊ~false��Ȼ����������һ�Ρ�

����pdf��ӡʱѡ�Page Scaling(ҳ�����)ѡ��none(��)�������ӡ������
���СһȦ���������ҳü��Ҳ�޷����롣

Ϊ�����ͳ�Ʋ��ҷ��㣬��ģ����������Ӣ��ͼ��Ŀ¼�����ǹ�������о������Ĺ淶�в�û������Ҫ��
���������Ҫ���뵽~main.tex �аѸöδ���ע�͵����ļ�����ע������

���İ���������Ȩ˵���飬�������Ѿ������ˡ������ȡ��һ����~appendix\textbackslash Authorization.tex ��ע�͵�һ���������ˡ�:-)

\BiSection{ͼ������Ӣ�ı���}{Chinese and English Caption of Figures and Tables}
\label{Tricks:Captions}
\BiSubsection{ͼ����}{Caption of Figures}
ģ����Ϊͼ������˫�������
\begin{verb}
\FigureBiCaption{����}{Ӣ��}
\end{verb}
�������������������һ��Ϊ���ı��⣬�ڶ���ΪӢ�ı��⡣
ͼ~\ref{Figure:Tricks:Example1}������һ����Ӣ�ı�������ӡ�

\begin{figure}[htbp]
\centering
\includegraphics[width = 0.4\textwidth]{golfer}
\FigureBiCaption{��߶��������}{Golfer}
\label{Figure:Tricks:Example1}
\end{figure}

���ijͼ��ܳ��Ļ�������ͨ���ֲ��ı�\verb|\captionwidth|�Ŀ��Ƚ��ж��С�
ͼ~\ref{Figure:Tricks:Example11}������һ����Ӣ�ı�����������ӡ�

\begin{figure}[htbp]
\centering
\includegraphics[width = 0.4\textwidth]{golfer}
\changecaptionwidth \captionwidth{0.7\textwidth}
\FigureBiCaption{һ����߶�������˴� �߶�������˴�߶���
�߶�������� ��߶� ������˴�߶����� ���˴�߶������
�˴�߶��������}{Golfer Golfer This is a very good idea and i
like it very much do u like it Golfer Golfer Golfer Golfer Golfer
Golfer Golfer Golfer Golfer} \label{Figure:Tricks:Example11}
\end{figure}

\normalcaptionwidth
ͼ~\ref{Figure:Tricks:Example12}�ǻָ�Ĭ�Ͽ���֮���ͼ�����ӡ�
\begin{figure}[htbp]
\centering
\includegraphics[width = 0.4\textwidth]{golfer}
\FigureBiCaption{��߶�������˴�߶�������˴�߶���߶�������˴�߶�������˴�߶�������˴�߶�������˴�߶��������}{Golfer Golfer Golfer Golfer Golfer Golfer   Golfer Golfer Golfer Golfer Golfer Golfer Golfer Golfer Golfer Golfer Golfer Golfer}
\label{Figure:Tricks:Example12}
\end{figure}


Ϊ��ͼ������һ��Ӣ�ı������
\begin{verb}
\SubfigEnCaption{Ӣ��}��
\end{verb}
�ڽ�����~subfigure~�������������,������Ӣ�ı��⡣
��һ�в�ֻһ����ͼʱ,��ͼ����~minipage~�У���~minipage~����������

ͼ~\ref{Figure:Tricks:Example2} ������һ��ֻ��һ����ͼ�����ӡ�

ͼ~\ref{Figure:Tricks:Example3} ������һ���ж����ͼ�����ӡ�
\begin{figure}[htbp]
\centering
\subfigure[�߶���]{\label{Figure:Tricks:Example2:a}
  \includegraphics[width = 0.22\textwidth]{golfer}
}\SubfigEnCaption{Golfer1}
\subfigure[�߶���]{\label{Figure:Tricks:Example2:B}
  \includegraphics[width = 0.22\textwidth]{golfer}
}\SubfigEnCaption{Golfer2 sina test long}
\FigureBiCaption{�߶���}{Golf}
\label{Figure:Tricks:Example2}
\end{figure}

\begin{figure}[htbp]
\centering
\begin{minipage}{0.25\textwidth}
\centering
\subfigure[�߶���]{\label{Figure:Tricks:Example3:a}
  \includegraphics[width = \textwidth]{golfer}
}\SubfigEnCaption{Golfer1}
\end{minipage}
\begin{minipage}{0.25\textwidth}
\centering
\subfigure[�߶���]{\label{Figure:Tricks:Example3:B}
  \includegraphics[width = \textwidth]{golfer}
}\SubfigEnCaption{Golfer2}
\end{minipage}
\begin{minipage}{0.25\textwidth}
\centering
\subfigure[�߶���3]{\label{Figure:Tricks:Example3:C}
  \includegraphics[width = \textwidth]{golfer}
}\SubfigEnCaption{Golfer3}
\end{minipage}
\FigureBiCaption{�߶���}{Golf}
\label{Figure:Tricks:Example3}
\end{figure}

ͼ~\ref{Figure:Tricks:Example31} ������һ����ͼͼ�����������
\begin{figure}[htbp]
\centering
\begin{minipage}[t]{0.20\textwidth}
\centering
\subfigure[�߶���߶���߶���߶���]{\label{Figure:Tricks:Example31:a}
  \includegraphics[width = \textwidth]{golfer}
}\SubfigEnCaption{Golfer1 Golfer1 Golfer1 Golfer1 Golfer1}
\end{minipage}\hspace{2em}
\begin{minipage}[t]{0.20\textwidth}
\centering
\subfigure[�߶���߶���߶���߷�]{\label{Figure:Tricks:Example31:B}
  \includegraphics[width = \textwidth]{golfer}
}\SubfigEnCaption{G G f f f f f f f f f f f f f f f f f f}
\end{minipage}\hspace{2em}
\begin{minipage}[t]{0.20\textwidth}
\centering
\subfigure[�߶���߶���߶���߶�3]{\label{Figure:Tricks:Example31:C}
  \includegraphics[width = \textwidth]{golfer}
}\SubfigEnCaption{Golfer3 Golfer3 Golfer3 Golfer3 Golfer3 Golfer3}
\end{minipage}
\FigureBiCaption{��ͼ���߶��� �߶��� �߶��� �߶��� �߶���}{Parent: Golf Golf Golf Golf Golf}
\label{Figure:Tricks:Example31}
\end{figure}


\BiSubsection{������}{Caption of Tables}
ģ���зֱ�Ϊ��������˫�������
\verb"\TableBiCaption{����}{Ӣ��}"��
�������������������һ��Ϊ���ı��⣬�ڶ���ΪӢ�ı��⣬Ч����~\ref{Tricks:Tab1}��

\begin{table}[htbp]
\centering
\TableBiCaption{�������}{Test of Table}
\label{Tricks:Tab1}
\begin{tabular}{c|c|c}
  \hline
  % after \\: \hline or \cline{col1-col2} \cline{col3-col4} ...
  ���� & ����~(\%) & �ٶ�~(ms) \\
  \hline
  С���任 & $99.8$ &  20\\
  ����Ҷ�任 & $99.0$ & 30 \\
  \hline
\end{tabular}
\end{table}

���ڳ��������Ӣ�ı��⣬����~\verb"\LTBiTocCaption{���ij����̱��⣨Ŀ¼�У�}{���ij��������⣨�����У�}{Table}{Ӣ�ij����̱��⣨Ŀ¼�У�}{Ӣ�ij��������⣨���ģ�}"
~�����塣��Ҫע����ǣ�����������~\verb|ccaption| ����еij�������⴦����ʽ������5��������
��������������ı䣬���������������ƣ����������׼��٣�������˶��塣
����������������ı��������ĺ�ͼ��Ŀ¼���������Ű棬���Һ���������ı��Э����
�����Ҫ���ó�������~\verb|\LTBiTocCaption|����֮�����~\verb|\label| ���
��~\ref{table:LTexample} ����һ�����ӡ�

\begin{longtable}{lll}
\LTBiTocCaption{���ı����}{���ı��ⳤ}{Table}{Long Table  Short Caption}{Long Table Long Caption}\label{table:LTexample}\\
\bfseries Entity & \bfseries Unicode Name & \bfseries Unicode \\ \hline
\endfirsthead
\bfseries Entity & \bfseries Unicode Name & \bfseries Unicode \\ \hline
\endhead
\hline \multicolumn{3}{r}{\emph{Continued on next page}}
\endfoot
\hline
\endlastfoot
a&emf&bcdef\\
a&emf&bcdef\\
a&emf&bcdef\\
a&emf&bcdef\\
a&emf&bcdef\\
a&emf&bcdef\\
a&emf&bcdef\\
a&emf&bcdef\\
a&emf&bcdef\\
a&emf&bcdef\\
a&emf&bcdef\\
a&emf&bcdef\\
a&emf&bcdef\\
a&emf&bcdef\\
a&emf&bcdef\\
a&emf&bcdef\\
a&emf&bcdef\\
a&emf&bcdef\\
a&emf&bcdef\\
a&emf&bcdef\\
%a&emf&bcdef\\
\end{longtable}

\BiSection{��ʽ}{Equations}
\label{Tricks:Equations}

�ı��е���ѧ���ź͹�ʽ������ķ������룺

������ѧ��������ȡ��һ���������ģ�;���ͨ����˵��$N$�����⣬����һ
�������£����о������屻�����ʵ㣬$N$��������򵥵ľ��Ƕ������⡣��һ
������ϵͳ�У�$N$��������������$n$���������$k$��С����($N=n+k$)������
$k$��С�������$n$�����������С���Ժ����˶���Ӱ�켸�����ÿ��ǣ���$k$
��С����֮��������Ͻ�������֮����໥����Ӧ�迼�ǣ���͹���������
��($n+k$)�����⡣�ر�أ���$N=3,~n=2,~k=1$ʱ����ͨ����˵���������������⡣


���������ѧ��ʽ������ŵģ�
\begin{equation}
\ddot{\mathbf{r}}=\mathbf{F}_{0}(r)+\mathbf{F}_{\varepsilon}(\mathbf{r},\dot{\mathbf{r}},t)
\end{equation}

����һ��������ŵ����ӣ�
\begin{displaymath}
F_{\varepsilon}/F_{0}=O (\varepsilon)
\end{displaymath}

\FloatBarrier %���������
���͵Ĺ�ʽ�ӷ���˵�������ӣ�

Ŀ���������׷�ٷ�����֮�������˶�����Ϊ��
\begin{equation}\label{eq:1}
\ddot{\boldsymbol{\rho}}-\frac{\mu}{R_{t}^{3}}\left( 3\mathbf{R_{t}}\frac{\mathbf{R_{t}\rho}}{R_{t}^{2}}-\boldsymbol{\rho}\right)=\mathbf{a}
\end{equation}
����

$\boldsymbol{\rho}$---׷�ٷ�������Ŀ�������֮������λ��ʸ����

$\ddot{\boldsymbol{\rho}}$---׷�ٷ�������Ŀ�������֮�����Լ��ٶȣ�

$\mathbf{a}$---�����������ļ��ٶȣ�

$\mathbf{R}_{t}$---Ŀ��������ڹ�������ϵ�е�λ��ʸ����

$\omega_{t}$---Ŀ��������Ĺ�����ٶȣ�

$\mathbf{g}=\frac{\mu}{R_{t}^{3}}\left(
3\mathbf{R_{t}}\frac{\mathbf{R_{t}\rho}}{R_{t}^{2}}-\boldsymbol{\rho}\right)=\omega_{t}^{2}\frac{R_{t}}{p}\left(
3\mathbf{R_{t}}\frac{\mathbf{R_{t}\rho}}{R_{t}^{2}}-\boldsymbol{\rho}\right)$---�������ٶȣ�����$p$��Ŀ��������Ĺ����ͨ����

��ʽ�ӷ���˵��������������
\begin{equation}\label{eq:111}
\ddot{\boldsymbol{\rho}}-\frac{\mu}{R_{t}^{3}}\left( 3\mathbf{R_{t}}\frac{\mathbf{R_{t}\rho}}{R_{t}^{2}}-\boldsymbol{\rho}\right)=\mathbf{a}
\end{equation}
\begin{formulasymb}{ʽ��}{-15pt}%-3pt,-20pt�����Ϸ��ļ�ࡣ
  \fdesfirst{$\boldsymbol{\rho}$}{׷�ٷ�������Ŀ�������֮������λ��ʸ����}
  \fdes{$\ddot{\boldsymbol{\rho}}$}{׷�ٷ�������Ŀ�������֮�����Լ��ٶȣ�}
  \fdes{$\mathbf{a}$}{�����������ļ��ٶȣ�}
  \fdes{$\mathbf{R}_{t}$}{Ŀ��������ڹ�������ϵ�е�λ��ʸ����}
  \fdes{$\omega_{t}$}{Ŀ��������Ĺ�����ٶȣ�}
  \fdes{$\mathbf{g}=\frac{\mu}{R_{t}^{3}}\left(
3\mathbf{R_{t}}\frac{\mathbf{R_{t}\rho}}{R_{t}^{2}}-\boldsymbol{\rho}\right)=\omega_{t}^{2}\frac{R_{t}}{p}\left(
3\mathbf{R_{t}}\frac{\mathbf{R_{t}\rho}}{R_{t}^{2}}-\boldsymbol{\rho}\right)$}{�������ٶȣ�����$p$��Ŀ��������Ĺ����ͨ����}
\end{formulasymb}

���о�����������Ĺ�ʽ��
\begin{equation}\label{eq:rho}
\dot{\boldsymbol{\rho}}=\left( \begin{array}{c}
\dot{x}-\omega_{t}y\\\dot{y}+\omega_{t}x\\\dot{z}
\end{array}\right) , \quad
\ddot{\boldsymbol{\rho}}=\left( \begin{array}{c}
\ddot{x}-2\omega_{t}\dot{y}-\omega_{t}^{2}x-\dot{\omega}_{t}y\\
\ddot{y}+2\omega_{t}\dot{x}-\omega_{t}^{2}y+\dot{\omega}_{t}x\\
\ddot{z}
\end{array}\right)
\end{equation}

���ڷֿ���󣬿��Բ���~arydshln~�������ģ�����Ѿ�������һ�����֧�֣�ʹ��ʱ������������ոú���Դ����ĵ���

\begin{equation}
A = \left\{ \begin{array}{cc:c} %: ��ʾ�������ߣ�
 x & y & z \\
 u & v & w \\ \hdashline   %�������
 x & y & z
 \end{array} \right\}
 \end{equation}


���һ��д�����������У���

{\setlength\arraycolsep{2pt}
\begin{eqnarray}
x & = & \left( x_{0}+\frac{2\dot{y}_{0}}{\omega_{t}}+\frac{4a_{x}}{\omega_{t}^{2}}\right) +2\left( \frac{2\dot{x}_{0}}{\omega_{t}}-3y_{0}-\frac{a_{y}}{\omega_{t}^{2}}\right) \sin (\omega_{t} t) \nonumber\\
& & -2\left( \frac{\dot{y}_{0}}{\omega_{t}}+\frac{2a_{x}}{\omega_{t}^{2}}\right) \cos (\omega_{t} t)-\left( 3\dot{x}_{0}-6\omega_{t} y_{0}-\frac{2a_{y}}{\omega_{t}}\right) t-\frac{3a_{x}}{2}t^{2} \\
y & = & \left( 4y_{0}-\frac{2\dot{x}_{0}}{\omega_{t}}+\frac{a_{y}}{\omega_{t}^{2}} \right) +\left( \frac{\dot{y}_{0}}{\omega_{t}}+\frac{2a_{x}}{\omega_{t}^{2}}\right) \sin (\omega_{t} t) \nonumber\\
& & -\left( 3y_{0}-\frac{2\dot{x}_{0}}{\omega_{t}}+\frac{a_{y}}{\omega_{t}^{2}}\right) \cos (\omega_{t}t)-\frac{2a_{x}}{\omega_{t}}t \\
z & = & \frac{\dot{z}_{0}}{\omega_{t}}\sin (\omega_{t}t)+\left( z_{0}-\frac{a_{x}}{\omega_{t}^{2}}\right)\cos (\omega_{t}t)+\frac{a_{z}}{\omega_{t}^{2}}
\end{eqnarray}}


�������������ʽʱ����Ҫÿ����ʽ����~equation~������������ʹ�ù�ʽ֮��ľ���ܴ�
�Ƽ�ʹ��~align~�����������뿴��Ӧ���ĵ���

����ͨ��~\verb"\setlength\jot{����}"~���趨��ʽ֮��ľ��룬Ĭ��Ϊ~3pt����ģ�彫���趨Ϊ~2.5ex��
\begin{gather}
\alpha + \beta = \gamma\\
x^2+y^2=z^2\\
E=mc^2
\end{gather}

\BiSection{һ������С�ڱ���һ������С�ڱ���һ������С�ڱ���һ������С�ڱ���һ������С�ڱ���һ������С�ڱ���}{A Long Section Title Example}\label{tricks:Longsectiontitle}

��\ref{tricks:Longsectiontitle}��һ���ڱ�����������ӡ��±������������Ҳ�Ѿ���������ﲻ�ٸ������ӡ�

\defaultfont

\BiChapter{ģ���������޸ļ�¼}{Update Record of the Thesis Model}
\label{Updatelog}

\BiSection{˵��}{Introduction}
\label{Update:intro}
Ϊ�˸�����Ч��ά��������ģ�壬�����Ӵ��£����Լ�¼ģ���������ĸĶ���
ͬʱ����Ҳ�������û���������˽��ģ�塣

Ϊ���ø����ͬѧ���������µ�����ģ�壬��������ʹ��ģ��ʱ�����ģ��
���κθĶ����߽��飬�������˵��϶���BBS��TeX����Լ����ֻ�������
����һ�¡�

���µļ�¼�����汾������bug�޸����κ��漰��ģ�����ݵĸĶ���

Ŀǰ����~\url{http://gf.cs.hit.edu.cn}~�ϴ�����~Pluto��ڤ���ǣ���������
ҵ��ѧѧλ����ģ�忪Դ��Ŀ����ҵ��޸Ŀ��Լ��е����

%%%%%%%%%%%%%%%%%%%%%%%%%%%%%%%%%%%%%%%%%%%%%%%%%%%%%%%%%%%%%%%%%%%%%%%%%%%%%
\BiSection{ģ��ĵ���}{The Naissance of the Template}
\label{version-1.0}
��ģ��������UFO��(2004)�����廪��ѧ��ʿ����ģ�壬���չ�������ҵ��ѧ��
�Ĺ淶������\LaTeX{}����ģ�塣����nebula��polar���������˶ʿ���ĵĹ���
�����޸ĵõ���˶ʿ����ģ�塣


%%%%%%%%%%%%%%%%%%%%%%%%%%%%%%%%%%%%%%%%%%%%%%%%%%%%%%%%%%%%%%%%%%%%%%%%%%%%%
\BiSection{�汾������1.1~(by nebula)}{Version Update(by nebula)}
\label{version-1.1}
�����linux+TeXlive�����¿���������ǩ��������⣬��л������ѧ��Huskier
���ѷ��ָ����Ⲣ�ṩ�˽����������лˮľ�廪����snoopyzhao�ṩ��gbk2uni
������롣

ģ��ĸĶ����£�
\begin{enumerate}
\item ������һ��Ŀ¼~tools�������������ļ���������������Դ�ļ���
gbk2uni�ǿ�ִ���ļ������뻷����gcc 3.2.2�������������
�������б��룻
\item �Ķ���makefile�ļ���
\item �Ķ���main.tex�ļ���
\item �Ķ��˱��ļ���
\end{enumerate}


%%%%%%%%%%%%%%%%%%%%%%%%%%%%%%%%%%%%%%%%%%%%%%%%%%%%%%%%%%%%%%%%%%%%%%%%%%%%%
\BiSection{�汾��˵��}{The Version Control on the template}
\begin{enumerate}
\item ģ������Pluto�ƻ�ʱ�İ汾Ϊ1.0�汾��
\item ����ûһ���޸İ汾�������ӣ�
\item ����������ģ���Ϊ3.0�汾��֮����``$\pi$''��ֵ��Ϊ�汾�ţ��Ժ�ÿ����һ�ξ�ȷ�Ƚ�һλ������
���\LaTeX{}�İ汾��¼����������������������
\end{enumerate}

\defaultfont

\BiChapter{д�������ģ��ά��������}{To Template Maintainers}

\BiSection{ģ��ά���򵥽���}{Simple Introduction about This
Template}

\BiSection{ά�����߽���}{Maintaining Tools Introduction}

% ��xelatex�����UTF8�ļ�������ÿ���ļ���ָ���ַ�����;
% main.tex���ֶ��ƶ���\atemp��\usewhat������
\ifx\atempxetex\usewhat 
\XeTeXinputencoding "gbk"
\fi
\defaultfont

%%%%%%%%%%%%%%%%%%%%%%%%%%%%%%%%%%%%%%%%%%%%%%%%%%%%%%%%%%%%%%%%%%%%%%%%%%%%%
\BiChapter{��Ȩ����}{Copyright Statement}

��ģ����ѭ~GPL~Э�顣���������������г���\\
UFO\\
cucme\\
Stanley\\
TeX\\
nebula\\
luckyfox\\
jdg\\
LaTeX\\
��������У��У�������ṩ�������ģ����������������ǡ�

\include{body/conclusion}   % 结论

%参考文�?
\defaultfont
\ifx\atempxetex\usewhat
\bibliographystyle{chinesebst_UTF8}
\else
\bibliographystyle{chinesebst_UTF8}
\fi
\addcontentsline{toc}{chapter}{\hei \ReferenceCName}      % 参考文献加入到中文目录
\addcontentsline{toe}{chapter}{\bfseries \ReferenceEName} % 参考文献加入到英文目录
\addtolength{\bibsep}{-0.8 em} \nocite{*}
\bibliography{reference/reference}

%\addtocontents{fen}{\protect\vskip1\baselineskip}
%\addtocontents{ten}{\protect\vskip1\baselineskip}
%英文图形和表格索引里加入空白行,通常放在 \include{appendix/appA}% 附录A之前�?
%区分开正文和附录的图形和表格,一般没有这个必要�?

\include{appendix/appA}            % 附录A
\defaultfont

\BiAppendixChapter{����˶ʿѧλ�ڼ�������������}{Papers Published in the Period of Master Education}

\begin{publist}
\item ����. ��Ŀ. �ڿ�. ��, ��(��): ҳ��

\item ����. ��Ŀ. �ڿ�. ��, ��(��): ҳ��

\item ����. ��Ŀ. �ڿ�. ��, ��(��): ҳ��
\end{publist}

%%% Local Variables: 
%%% mode: latex
%%% TeX-master: "../main"
%%% End: 
    % 所发文�?
    \newpage

    %\thispagestyle{empty}
    \BiAppendixChapter{��������ҵ��ѧ˶ʿѧλ����ԭ��������}{STATEMENT OF COPYRIGHT}
    ����֣���������˴����ύ��˶ʿѧλ���ġ��߼��������������ķ�չ�о���
    ���DZ����ڵ�ʦָ���£��ڹ�������ҵ��ѧ����˶ʿѧλ�ڼ���������о�������ȡ��
    �ijɹ����ݱ�����֪�������г���ע�������ⲻ���������ѷ�����׫д�����о��ɹ���
    �Ա��ĵ��о�����������Ҫ���׵ĸ��˺ͼ��壬��������������ȷ��ʽע������������
    ���ɽ������ȫ�ɱ��˳е���
    \vspace{0.5cm}
    \begin{center}{
    ����ǩ����~~~~~~~~~~~~~~~~~~~~~~~~~~~~~~~����:~~~~~~~~~~~��~~~~~��~~~~~��}

    \end{center}

\phantomsection
\addcontentsline{toc}{chapter}{\hei ��������ҵ��ѧ˶ʿѧλ����ʹ����Ȩ��}
\addcontentsline{toe}{chapter}{\bfseries LETTER OF AUTHORIZATION}

    \vspace{0.3cm}
    \begin{center}{\xiaoer \hei
                    ��������ҵ��ѧ˶ʿѧλ����ʹ����Ȩ��}
    \end{center}
    \vspace{0.4cm}

���߼��������������ķ�չ�о���ϵ�����ڹ�������ҵ��ѧ����˶ʿѧλ��
���ڵ�ʦָ������ɵ�˶ʿѧλ���ġ������ĵ��о��ɹ����������ҵ��ѧ���У���
���ĵ��о����ݲ�����������λ�����巢����������ȫ�˽��������ҵ��ѧ���ڱ���
��ʹ��ѧλ���ĵĹ涨��ͬ��ѧУ���������йز����ͽ����ĵĸ�ӡ���͵��Ӱ汾��
�������ı����ĺͽ��ġ�������Ȩ��������ҵ��ѧ�����Բ���Ӱӡ����ӡ����������
�ֶα������ģ����Թ������ĵ�ȫ���򲿷����ݡ�

\vspace{0.5cm}
~~~~~~~~~~~~~~~~~~~~~~~~~~~~~~~��~~~��$\square$����~~~~����ܺ����ñ���Ȩ�顣

��ѧλ��������

~~~~~~~~~~~~~~~~~~~~~~~~~~~~~~������ $\square$��

������������Ӧ�����ڴ�$\surd$���� \vspace{1.0cm}
\begin{center}{
����ǩ����~~~~~~~~~~~~~~~~~~~~~~~~~~~~~~~����:~~~~~~~~~~~��~~~~~��~~~~~��}
\end{center}
\vspace{0.2cm}
\begin{center}{
��ʦǩ����~~~~~~~~~~~~~~~~~~~~~~~~~~~~~~~����:~~~~~~~~~~~��~~~~~��~~~~~��}
\end{center}


   % 原创性申�?
\defaultfont

\BiAppendixChapter{��~~~~л}{ACKNOWLEDGEMENT}

���������ڵ�ʦ��������ڵ�Ϥ��ָ������ɵġ�����ʦ�ḻ������֪ʶ���Ͻ�����ѧ
�����������Եľ��壬����ʦ��ѧ���о��ϵ����񶴲����Ͳ��ϸ���ѧ��ǰ�صľ�ҵ����ֵ��
������ѧϰ������һֱָ���ͼ����ҽ��Ĺ��������

���о��������ѧϰ�ڼ䣬лѷ��ʦ������Ī��İ�����ָ����л��ʦ�Ͻ��Ŀ���
���磬�����˼ά�����ҵĹ�������͸߶ȵ�������ʹ�������dz��л��ʦ����ѧ
���ϵ�ָ�������ܹ��˷����ѣ����뵽�о������С�

�ڴ�ѧλ�������֮�ʣ�����
������ѧϰ�ϡ������ϸ�������Ϲػ�����˽��������������ں�лѷ��ʦ��ʾ�����Ե�
��л�;��⣡

���ʦ̫���ں�������ʦ���ڹ�������������Ҵ����İ�����ָ������ʹ�����ҽӴ�
���˵���������ǰ�ص����ۺͼ������ڴ�������ʦ������ʦ��ʾ���ĵĸ�л��


��л������������������������������������ͬѧ��ѧϰ�����Ĺ����ڼ���ҵ�������
������

��л�ҵļ��˳�������������˽�Ĺذ���֧�֣�
% 致谢
\include{appendix/Resume}          % 个人简�?

\clearpage
\ifx\atempxetex\usewhat\else
\end{CJK}
\fi

\end{document}
